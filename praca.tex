% !TeX spellcheck = <none>
\documentclass[magister]{dyplom}
\usepackage[utf8]{inputenc}
\usepackage{hyperref}
\usepackage{verbatim}
\usepackage{minted}
\usepackage{lmodern}
\usepackage{subfig}
\usepackage{icomma}
%%
\usepackage[toc]{appendix}
\renewcommand{\appendixtocname}{Dodatki}
\renewcommand{\appendixpagename}{Dodatki}
%%
\usepackage{listings}
\usepackage{pdfpages}
%\noappendicestocpagenum
\usepackage{lipsum}
\usepackage{url}
%\usepackage{showframe}
\def\code#1{\texttt{#1}}

\author{Radosław Majchrzak}\album{178640}
\title{Narzędzie komputerowe wyznaczające optymalny portfel dla inwestora – portfel Markowitza}
\titlen{}
\promotor{dr hab. inż. prof. Uczelni Zbigniew Michna}
\kierunek{Informatyka w Biznesie}
\specjalnosc{Big Data}
\date{}

\begin{document}
\includepdf[pages={1}]{titlepage.pdf}
\linespread{1.3}
\tableofcontents


\chapter{Cel i zakres pracy}

W roku 1952 amerykański ekonomista Harry Markowitz w artykule \textit{Portfolio Selection} opublikował jedną z najważniejszych teorii w świecie finansów, zwaną teorią portfelową, lub od nazwiska twórcy teorii - Portfelem Markowitza. Był to pierwszy model portfela inwestora, w którym został jawnie wprowadzony parametr ryzyka w inwestycji kapitału\cite{wikipage}. Pokrótce mówiąc, rozwiązanie zaproponowane przez autora artykułu ma maksymalizować zysk inwestora przy danym stopniu ryzyka lub minimalizować ryzyko przy danym stopniu zysku. Teoria ta stała się dla mnie na tyle interesująca, że zdecydowałem się na oparcie na niej mojej pracy magisterskiej.\\\par
Przedmiotem pracy jest stworzenie narzędzia do wyznaczania portfela styczności oraz portfela optymalnego dla danego inwestora (tzn. inwestora z określoną awersją do ryzyka przy wybranej stopie wolnej od ryzyka w jego portfelu oraz wybranych przez niego aktywów ryzykownych) i zbadanie działania tego narzędzia.
Będzie to program napisany w języku programowania \textit{Python} w wersji 3.7.8.\\\par
W pracy zawarta jest dokumentacja opisowa narzędzia (zwanego w dalszych częściach pracy \textit{programem} lub \textit{aplikacją}). Wszystkie szczegóły dotyczące rozwiązań technicznych oraz opis matematyczny teorii są umieszczone w odpowiednich rozdziałach i podrozdziałach.\\\par

\chapter{Wprowadzenie matematyczne}
W tym rozdziale zostały krótko wyjaśnione pojęcia matematyczne, które są użyte w tej pracy:\\
\begin{itemize}
\item \textit{zmienna losowa} - jest to funkcja przypisująca zdarzeniom elementarnym liczby\cite{random_variable}.
\newline

\item \textit{wartość oczekiwana} - dla zmiennej losowej X typu skokowego jest opisana przez liczbę:
\begin{equation}
	EX = \sum_{n=1}^{\infty}x_np_n,
\end{equation}
gdzie:\\
$p_n$ - rozkład zmiennej losowej ($p_n = P(X = x_n)$),\newline
$x_n$ - pojedyncza obserwacja zmiennej losowej X.

Estymatorem wartości oczekiwanej rozkładu zmiennej losowej w populacji jest średnia arytmetyczna $n$ niezależnych obserwacji danej zmiennej, opisana wzorem \cite{expected_value}:
\begin{equation}
	\overline{X} = \frac{1}{n}(X_1 + X_2 + X_3 + \dots + X_n),
\end{equation}
gdzie:\\
$X$ - zmienna będąca obiektem działania,\\
$X_n$ - pojedyncza obserwacja zmiennej.
\newline

\item \textit{wariancja} - miara zmienności danych, oznaczana w matematyce przez znak $\sigma^2$. Określa to, jak bardzo wszystkie możliwe wartości zmiennej są rozproszone względem jej \textit{wartości oczekiwanej}. Dla zmiennej losowej X typu skokowego jest opisana przez liczbę:

\begin{equation}
	D^2(X) = \sum_n(x_n - EX)^2p_n.
\end{equation}

Często używanym estymatorem jest wariancja próbkowa, gdzie wzór przedstawia się następująco:

\begin{equation}
	\sigma^2 = \frac{(X_1 - \overline{X})^2 + (X_2 - \overline{X})^2 + \dots + (X_n - \overline{X})^2}{n}.
\end{equation}

\item \textit{odchylenie standardowe} - pierwiastek kwadratowy z wariancji: \newline

\begin{equation}
	D(X) = \sqrt{D^2(X)},
\end{equation}

\begin{equation}
	\sigma = \sqrt{\frac{(X_1 - \overline{X})^2 + (X_2 - \overline{X})^2 + \dots + (X_n - \overline{X})^2}{n}}.
\end{equation}
\newpage
\item \textit{kowariacja} - miara relacji między dwiema \textit{zmiennymi losowymi} - może przyjmować dowolne wartości zarówno dodatnie, jak i ujemne, które mają następującą interpretację:
\begin{itemize}
	\item Wartość dodatnia kowariancji - dwie zmienne losowe mają tendencję do ruchu w tym samym kierunku (to znaczy - jak jedna zmienna przyjmuje duże wartości to i druga zmienna losowa przyjmuje duże wartości),
	\item Wartość ujemna kowariancji - dwie zmienne losowe mają tendencję do ruchu w tym przeciwnym kierunku (to znaczy - jak jedna zmienna przyjmuje małe wartości to i druga zmienna losowa przyjmuje małe wartości),
	\item Wartość kowariancji to 0 - dwie zmienne losowe są nieskorelowane.
\end{itemize}

Kowariancja dla próby jest opisana wzorem:
\begin{equation}
	Cov(X,Y) = \frac{\sum(X_i - \overline{X})(Y_j - \overline{Y})}{n - 1}.
\end{equation}

\item \textit{korelacja} - miara relacji między dwoma zmiennymi losowymi, wynikiem jest liczba należąca do zbioru $<-1,1>$. Korelacja dla próby jest opisana wzorem:
\begin{equation}
	\rho_{X,Y} = \frac{Cov(X,Y)}{\sigma_X\sigma_Y},
\end{equation}
gdzie:\\
$\sigma_X$ - odchylenie standardowe zmiennej losowej X,\\
$\sigma_Y$ - odchylenie standardowe zmiennej losowej Y.\\

Interpretacja wartości korelacji jest taka sama jak dla kowariancji. Im wartość korelacji jest bliższa wartości skrajnej, tym mocniejsze jest oddziaływanie jednej zmiennej na drugą.\\

\item \textit{rozkład prawdopodobieństwa} - jest to przedstawienie możliwych wartości zmiennej oraz tego, jak często mogą one wystąpić. Zbiór możliwych wartości jest opisany przez \textit{X}, a pojedyncza wartość z tego zbioru przez \textit{x}. Najczęściej występującym w przyrodzie rozkładem prawdopodobieństwa jest \textit{rozkład normalny}.\newline
\newpage
\item \textit{rozkład normalny prawdopodobieństwa} - najczęściej występujący w przyrodzie rozkład prawdopodobieństwa - wykres jest krzywą w kształcie dzwonu, widocznego na rysunku 2.1:
\begin{figure}[h]
	\centering
	\includegraphics[scale=0.4]{Standard_Normal_Distribution.png}
	\caption{Wykres rozkładu normalnego, źródło:\newline
		https://commons.wikimedia.org/wiki/File:Standard\_Normal\_Distribution.png.}
\end{figure}
\end{itemize}
\chapter{Teoria portfela Markowitza}

\section{Wprowadzenie}

Jak wiadomo, pojedyncze akcje na giełdzie cechują się dość dużą nieprzewidywalnością - ich zachowania można przewidywać, jednak zdarzenia losowe mogą takie spekulacje brutalnie zweryfikować. Jako przykład na rysunku 3.1 przedstawiono wykres ceny akcji spółki Nokia w ostatnich 2 latach. Jest on niejednostajny, zauważalne są duże spadki, które przypadały na zdarzenia takie jak choćby ogłoszenie zmiany prezesa firmy (trzeci kwartał 2019 roku) oraz początek pandemii koronawirusa (pierwszy kwartał 2020 roku). 

\begin{figure}[h]
\includegraphics[width=\textwidth]{nokia}
\caption{Ceny akcji firmy NOKIA(NOK1V) w okresie 21.01.2019-21.01.2021, źródło:\newline
	https://www.bankier.pl/inwestowanie/profile/quote.html?symbol=NOK1V.}
\end{figure}

Jak zauważa Markowitz w swoim artykule \textit{Portfolio Selection} - inwestor może uchronić się (w pewnym stopniu) przed niestabilnością rynku przez stworzenie portfela składającego się z różnych aktywów tak, by zminimalizować ryzyko\cite{markowitz}. Minimalizacja ryzyka przez zakup różnych aktywów nazywana jest \textit{dywersyfikacją}. Autor w swoim modelu przyjmuje dwa ważne założenia: 

\begin{enumerate}
	\item Zwroty mają rozkład normalny.
	\item Inwestorzy są racjonalni i niechętni do podejmowania ryzyka - mogą je podjąć tylko, jeżeli przewidują wyższy zwrot.
\end{enumerate}

Załóżmy, że portfel składa się z ustalonych aktywów. Stopa zwrotu z tego portfela po ustalonym czasie jest zmienną losową o pewnej wartości oczekiwanej i wariancji (zależnej od wag aktywów w tym portfelu).
Punkt drugi założeń Markowitza prowadzi więc do ważnej konkluzji - dla jakiegokolwiek akceptowalnego poziomu ryzyka istnieje tylko jeden portfel, który da najwyższy oczekiwany zwrot (dla oczekiwanych wag osiąga on maksimum) i dla danej średniej (oczekiwanej) stopy zwrotu istnieje tylko jeden portfel z minimalnym ryzykiem (dla odpowiednich wag osiąga ono minimum). Każdy taki portfel nazywany jest \textit{portfelem wydajnym} (lub \textit{efektywnym}).\par

\section{Portfele wydajne dla dwóch aktywów ryzykownych} 

Załóżmy, że na rynku dostępne są dwa walory ryzykowne. Nasz portfel składa się z dwóch takich aktywów, $S_1$ oraz $S_2$. Zmienne losowe $R_1$ oraz $R_2$ opisują stopę zwrotu z inwestycji w te aktywa po ustalonym okresie. Korelacja pomiędzy stopami zwrotu inwestycji wynosi $\rho$. Zadanie to znalezienie portfeli wydajnych - to znaczy takich proporcji tych dwóch walorów, dla których powinno się otrzymać największy średni zysk przy ustalonym ryzyku.\\
Oczekiwany zwrot z portfela po danym okresie będzie wynosić:

\begin{equation}
	R = w_1R_1 + w_2R_2,
\end{equation}
gdzie:\\
$w_n$ - waga aktywa n w portfelu,\\
$R$ - oczekiwany zwrot z całego portfela.\\
$R$, podobnie jak $R_1$ oraz $R_2$ jest zmienną losową. Mając na uwadze założenie Markowitza, że zwroty mają rozkład normalny, wspólny rozkład prawdopodobieństwa wszystkich tych zmiennych (tj. $R_1$ i $R_2$) jest dwuwymiarowym rozkładem normalnym - stąd są one całkowicie scharakteryzowane przez swoje wartości średniej ($\mu_n$) i odchylenia standardowego ($\sigma_n$). W takim przypadku wzór na oczekiwaną stopę zwrotu aktywów $S_1$ oraz $S_2$ opisuje równanie 3.2, a wariancję równanie 3.3:

\begin{equation}
	\mu = w_1\mu_1 + w_2\mu_2,
\end{equation}
gdzie:\\
$w_n$ - waga aktywa n w portfelu,\\
$\mu$ - średni zwrot z całego portfela.\\
\begin{equation}
	\sigma^2 = w_1^{2}\sigma_1^{2} + w_2^{2}\sigma_2^{2} + 2w_1\sigma_1w_2\sigma_2\rho, 
\end{equation}
gdzie:\\
$w_n$ - waga aktywa n w portfelu,\\
$\sigma^2$ - wariancja całego portfela.\\
Wagi walorów w portfelu muszą sumować się do 1. Wiedząc, że portfel składa się z dwóch walorów:
\begin{equation}
	w_1 + w_2 = 1,
\end{equation}
przy czym wartości wag, niezależnie od ilości aktywów nie są ograniczone do zakresu $<0,1>$ - ujemna waga waloru oznacza, że aktywo takie staje się w portfelu \textit{pozycją krótką}. Oznacza to, że pożyczamy to aktywo, sprzedajemy i inwestujemy wszystko co mamy w drugie aktywo.\\
Wzory 3.2 oraz 3.3 można uprościć, jako $w_2$ podstawiając $1 - w_1$, dzięki czemu pozbywamy się jednej niewiadomej. Rezultatem jest:

\begin{equation}
	\mu = w_1\mu_1 + (1 - w_1)\mu_2 ,
\end{equation}

\begin{equation}
	\sigma^2 = w_1^{2}\sigma_1^{2} + (1 - w_1)^{2}\sigma_2^{2} + 2w_1\sigma_1(1 - w_1)\sigma_2\rho.
\end{equation}
\\
Wyznaczając wagę $w_1$ z równania 3.5, otrzymujemy:

\begin{equation}
	w_1 = \frac{\mu - \mu_2}{\mu_1 - \mu_2},
\end{equation}
a podstawiając otrzymaną w równaniu 3.7 wagę $w_1$ do równania 3.6 otrzymujemy:

\begin{equation}
	\sigma^2 = \frac{(\mu - \mu_2)^2}{(\mu_1 - \mu_2)^2}\sigma_1^{2} + 2\frac{(\mu - \mu_2)(\mu_1 - \mu)}{(\mu_1 - \mu_2)^2}\rho\sigma_1\sigma_2 + \frac{(\mu_1 - \mu)^2}{(\mu_1 - \mu_2)^2}\sigma_2^{2}.
\end{equation}
Zauważmy, że zmienne $\sigma$ oraz $\mu$ zmieniają się ze zmianą wagi $w_1$, zaś zmienne związane z walorami (tzn. $\mu_1$, $\sigma_1$, $\mu_2$, $\sigma_2$, $\rho$) są niezależne od wag walorów. Równanie 3.8 można więc przekształcić w taki sposób, by utworzyć współczynnik $A$ dla zmiennych związanych z walorami:

\begin{equation}
	\sigma^2  = A(\mu - \mu_0)^2 + \sigma_0^2,
\end{equation}
gdzie:

\begin{equation}
	A = \frac{\sigma_1^{2} - 2\sigma_1\sigma_2\rho + \sigma_2^{2}}{(\mu_1 - \mu_2)^2} > 0,
\end{equation}
a $\sigma_0$ oraz $\mu_0$ to współrzędne określające wierzchołek hiperboli 3.10:

\begin{equation}
	\sigma_0^2 = \frac{\sigma_1^{2}\sigma_2^{2}(1-\rho^2)}{(\sigma_1 - \sigma_2)^2 + 2(1-\rho)\sigma_1\sigma_2} \ge 0,
\end{equation}

\begin{equation}
	\mu_0 = \frac{\mu_1\sigma_2^2 - (\mu_1 + \mu_2)\rho\sigma_1\sigma_2 + \mu_2\sigma_1^2}{(\sigma_1 - \sigma_2)^2 + 2(1-\rho)\sigma_1\sigma_2} \ge 0,
\end{equation}
gdzie w powyższych równaniach nierówność wynika z tego, że $ -1 < \rho < 1$. \cite{book}\\

Wykres krzywej opisanej równaniem 3.9 przedstawiony został na rysunku 3.2.
Otrzymano typowy kształt wyznaczający wcześniej wspomniane portfele wydajne - są to portfele z górnej części hiperboli. Zgodnie z teorią Markowitza dają one najwyższy spodziewany zwrot przy danym ryzyku. Portfele, których punkty znajdują się w dolnej części hiperboli nazywane są \textit{portfelami niewydajnymi} i nie są rozpatrywane w wyborze portfela. Jest to widoczne na rysunku 3.2 - ich oczekiwany zwrot jest niższy niż tych z górnej hiperboli przy jednakowym ryzyku.\\

Dla łatwiejszgo zrozumienia, oto przykład portfela będącego tematem tego podrodziału (to znaczy składającego się z dwóch aktywów ryzykownych). Zwrot portfela jest opisany zmienną losową $R$, a stopę zwrotu z inwestycji w aktywa po ustalonym okresie opisują zmienne $R_1$ oraz $R_2$. Wartości oczekiwane i odchylenia standardowe $R_1$ i $R_2$ wynoszą odpowiednio: $\mu_1 = 7\%$, $\sigma_1 = 5\%$, $\mu_2 = 12\%$, $\sigma_2 = 7\%$, a ich korelacja $\rho = 0,2$.\\
Aby znaleźć wierzchołek hiperboli opisującej ten portfel, podstawiamy dane odopwiednio do wzorów 3.11 i 3.12:
\begin{equation}
	\mu_0 = \frac{102}{12} = 8,5,
\end{equation}

\begin{equation}
	\sigma_0 = \sqrt{\sigma_0^2} = \sqrt{\frac{1176}{60}} \approx 4,43.
\end{equation}
Wierzchołek hiperboli znajduje się więc w punkcie o współrzędnych $(4,43; 8,5)$. Dane zaznaczono na rysunku 3.2:\\
\begin{figure}[h]
	\centering
	\includegraphics[scale=0.27]{wykres_markowitz_z_wierzcholkiem}
	\caption{Zależność odchylenia standardowego portfela i jego oczekiwanego zwrotu z zaznaczonym wierzchołkiem hiperboli.}
\end{figure}

Obliczenie wag aktywów dla tego przypadku jest teraz bardzo proste - podstawiając do wzoru z równania 3.7 wyliczony oczekiwany zwrot $\mu$ otrzymujemy wagę $w_1$, a do znalezienia $w_2$ korzystamy z równania 3.4:

\begin{equation}
	w_1 = \frac{8,5 - 12}{7 - 12} = 0,7,
\end{equation}

\begin{equation}
	w_2 = 1 - 0,7 = 0,3.
\end{equation}
Wynik ten oznacza, że jeżeli nasz budżet wynosi 1000 PLN powinniśmy zainwestować 700 PLN w aktywo 1, a 300 w aktywo 2, aby otrzymać maksymalny oczekiwany zwrot dla ryzyka $\sigma = 4,43$.\\

Jak wcześniej wspomniano, możliwy jest również scenariusz w którym jedno z aktywów będzie miało wagę ujemną i oznacza to, że powinniśmy takie aktywo sprzedać i za to i za kwotę, którą posiadamy kupić drugie aktywo. Jako że mamy ustalony budżet, sprzedaż taka jest w niego wliczona - tzn. mając 1000 PLN i wagi $w_1 = 1,2$ i $w_2 = -0,2$, odpowiedź brzmi: "Sprzedaj aktywo 2 o wartości 200 PLN i zainwestuj 1200 PLN w aktywo 1".\\

Inwestor może korzystać z różnych portfeli, balansując ryzykiem i stopą zwrotu, jednak trzeba pamiętać o założeniu Markowitza - inwestor jest racjonalny, a więc jego wybór ograniczony jest tylko i wyłącznie do portfeli z górnej części hiperboli, ponieważ dają one największy średni zysk przy danym ryzyku.

\section{Portfel optymalny dla danego inwestora}
W rozważaniach przyjęto portfel składający się z dwóch aktywów ryzykownych. Jak wiadomo, każdy człowiek jest inny - dlatego każdy inwestor ma pewną, określoną przez siebie tolerację na ryzyko wynikającą z różnych czynników. W teorii Markowitza ta tolerancja inwestora na ryzyko określana jest przez \textit{funkcję użyteczności} U. Jest to funkcja, której dziedziną są rozkłady stopy zwrotów. Zmienna losowa $R$ opisująca stopę zwrotu po danym okresie ma rozkład $F_R$. Funkcja użyteczności ma postać widoczną na równaniu 3.17:
\begin{equation}
U(F_R) = \mu - k\sigma^2,
\end{equation}
gdzie:\\
$\mu$ - średnia dystrybuanty $F_R$,\\
$\sigma$ - odchylenie standardowe dystrybuanty $F_R$,\\
$k$ - współczynnik awersji do ryzyka (indywidualny dla każdego inwestora).\\
Jak możemy zauważyć, do obliczeń potrzebne są tylko wartości $\mu$ oraz $\sigma$ tego rozkładu. Aby uzyskać portfel optymalny dla danego inwestora, należy znaleźć taki portfel wydajny (czyli, tak jak już wiadomo, leżący na górnej części hiperboli), który maksymalizuje funkcję użyteczności awersji do ryzyka.\\
W sensie analitycznym rozwiązanie przedstawia się następująco: szukamy maksimum funkcji U przy warunku 3.9 ograniczając się do górnej hiperboli tego równania, gdyż jak zostało wcześniej powiedziane - tylko te portfele są portfelami wydajnymi. Można pokazać, że szukany portfel optymalny to portfel wyznaczony przez punkt styczności funkcji użyteczności i górnej części hiperboli - nazwijmy go punktem $O$. Graficznie zostało to przedstawione na rysunku 3.3:\\
\newpage
\begin{figure}[h]
	\centering
	\includegraphics[scale=0.4]{wykres_markowitz_z_u}
	\caption{Wykres hiperboli wyznaczającej portfele oraz paraboli funkcji użyteczności U z zaznaczonym ich punktem wspólnym $O$.}
\end{figure}

Zadaniem jest znalezienie współrzędnych tego punktu. Aby nie używać oznaczeń $(\sigma_o, \mu_o)$, które mogą być mylące (O jest zbyt podobne do 0), użyto $(\sigma^*, \mu^*)$. Podstawiając $\sigma^2$ z równania 3.9 do równania 3.16:

\begin{equation}
	U = \mu^* - kA(\mu^* - \mu_0)^2 + k\sigma_0^2.
\end{equation}
Obliczany punkt maksymalny to taki, gdzie pochodna cząstkowa $\frac{\partial\\U}{\partial\mu^*} = 0$, więc:

\begin{equation}
	\frac{\partial\\U}{\partial\mu^*} = 1 - 2kA(\mu^* - \mu_0) = 0.
\end{equation}
Przekształćmy równanie, aby po jego lewej stronie znalazł się jedynie szukany oczekiwany zwrot portfela $\mu^*$:

\begin{equation}
	\mu^* = \mu_0 + \frac{1}{2Ak}.
\end{equation}
Odchylenie standardowe tego portfela obliczamy, podstawiając znalezione $\mu^*$ do równania hiperboli 3.9:

\begin{equation}
	\sigma^* = \sqrt{{\sigma_0}^2+\frac{1}{4Ak^2}}.
\end{equation}
Wyliczony z równań 3.20 i 3.21 punkt $(\sigma^*, \mu^* )$ stanowi punkt portfela optymalnego danego inwestora. Wagi poszczególnych aktywów dla tego portfela wyznacza się w analogiczny sposób, jak w sekcji 3.2 (korzystając z równania 3.9), oczywiście za $\mu$ przyjmując oczekiwany zwrot portfela optymalnego, czyli $\mu^*$.

\section{Portfele wydajne dla aktywa wolnego od ryzyka i jednego aktywa ryzykownego}
W portfelu może znajdować się \textit{aktywo wolne od ryzyka}. Jest to takie aktywo, którego stopa zwrotu przyjmuje jedną wartość z prawdopodobieństwem równym 1, a więc jego wariancja wynosi 0. Inwestor wybiera tylko jedno aktywo wolne od ryzyka, gdyż z założenia Markowitza o jego racjonalności, będzie wybierać te o najwyższej stopie zwrotu. Jak będzie wyglądać portfel wydajny zawierający takie aktywo? Załóżmy, że mamy portfel z dwoma aktywami, $S_1$ oraz $S_2$, gdzie $S_1$ jest aktywem ryzykownym, a $S_2$ wolnym od ryzyka. $R_1$ oraz $R_2$ to zmienne losowe opisujące stopę zwrotu z inwestycji w te aktywa, i są one opisane przez ich odchylenia standardowe $\sigma_1$ i $\sigma_2 = 0$ oraz wartości średnie $\mu_1$ i $r_f$. Znając wzór na odchylenie standardowe portfela (równanie 3.6) znajdźmy równanie opisujące oczekiwany zwrot z opisywanego portfela:

\begin{equation}
\sigma^2 = w_1^{2}\sigma_1^{2}.
\end{equation}
Podstawmy w miejsce $w_1$ wartość wyznaczoną w równaniu 3.7:

\begin{equation}
	\sigma = \frac{\mu - r_f}{\mu_1 - r_f}\sigma_1,
\end{equation}
po przekształceniu, aby po jego lewej stronie znalazł się jedynie szukany oczekiwany zwrot portfela $\mu$ otrzymujemy:

\begin{equation}
	\mu = r_f + \frac{\mu_1 - r_f}{\sigma_1}\sigma.
\end{equation}
Równanie 3.23, po przeniesieniu na wykres i podstawieniu danych ukazano na rysunku 3.4:

\begin{figure}[h!]
	\centering
	\includegraphics[scale=0.27]{zero_risk_wykres}
	\caption{Zależność odchylenia standardowego portfela i jego oczekiwanego zwrotu dla portfela zawierającego aktywo ryzykowne ($\sigma_1, \mu_1$) oraz wolne od ryzyka ($0, r_f$).}
\end{figure}
\newpage
Podsumowując - otrzymujemy, że portfele wydajne dla 2 aktywów, z których jedno jest wolne od ryzyka leżą na linii prostej przechodzącej przez współrzędne $(\sigma, \mu)$ tych aktywów.

\section{Portfele wydajne dla aktywa wolnego od ryzyka i dwóch aktywów ryzykownych}
W rozważaniach przyjęto portfel składający się z dwóch aktywów ryzykownych $S_1$ oraz $S_2$ oraz aktywa wolnego od ryzyka $S_3$. Zmienne losowe $R_1$, $R_2$ oraz $R_3$ opisują stopę zwrotu z inwestycji w te aktywa po ustalonym okresie. Zmienne $R_1$ oraz $R_2$ mają rozkład normalny i są opisane przez ich średnie $\mu$ i odchylenia standardowe $\sigma$. $R_3$ jest opisane przez stopę zwrotu $r_f$, a odchylenie standardowe tej zmiennej wynosi $\sigma_f = 0$, co wiemy z poprzedniego podrozdziału. Przyjmujemy również, że jedno z aktywów ma wyższą zarówno stopę zwrotu, jak i odchylenie standardowe ($\mu_1 < \mu_2$ oraz $\sigma_1 < \sigma_2$). Przykładowo, w sytuacji gdy $\mu_1 < \mu_2$ oraz $\sigma_1 > \sigma_2$ oczywistym byłby wybór aktywa opisanego przez zmienną losową $R_1$ - dawałoby ono większy oczekiwany zysk przy mniejszym ryzyku, a więc inwestowalibyśmy tylko w aktywo 2.\\
Wiemy, że portfele wydajne zawierające dwa aktywa ryzykowne leżą na górnej części hiperboli opisanej równaniem 3.9 oraz to, że portfele zawierające aktywo wolne od ryzyka i aktywo ryzykowne leżą na linii prostej opisanej równaniem 3.24. Opis graficzny z przykładem znajduje się na rysunku 3.5:\\ 
\newpage
\begin{figure}[h]
	\centering
	\includegraphics[scale=0.4]{wykres_markowitz_proste_wydajne}
	\caption{Wykres hiperboli opisującej portfele aktywów ryzykownych oraz prostych opisujących portfele zawierające aktywo wolne od ryzyka.}
\end{figure}

Jak można zauważyć na rysunku 3.5, portfel leżący na prostej przechodzącej przez punkty $E$ oraz $(0, r_f)$ dla przykładowych danych ryzyk $\sigma_i$ oraz $\sigma_j$ daje wyższą stopę zwrotu (punkty $I$ oraz $J$) niż portfel leżący na prostej przechodzącej przez punkty $N$ oraz $(0, r_f)$ (punkty $I'$ oraz $J'$). Najlepszym portfelem - takim, który spełnia założenia Markowitza o maksymalizacji oczekiwanego zwrotu oraz minimalizacji ryzyka - będzie więc portfel leżący na prostej stycznej do hiperboli, a więc zawierającej punkt $E$. Portfel z punktu $E$, w tym wypadku składujący się z 2 aktywów ryzykownych $S_1$ i $S_2$, nazywamy \textit{portfelem styczności}. Portfele wydajne leżą na prostej przechodzącej przez punkt $(0,r_f)$ i punkt $E$, tak więc składają się z aktywa wolnego od ryzyka i portfela $E$. Stąd każdy inwestor niezależnie od swojej awersji do ryzyka ma taki sam portfel aktywów ryzykownych, portfele różnią jedynie się częścią zainwestowaną w aktywo wolne od ryzyka i portfel styczności $E$. Nazywamy to \textit{twierdzeniem o wzajemnej separacji funduszy} (ang. \textit{mutual fund separation theorem})\\ 

Było to odkrycie, które wywołało szeroką dyskusję w świecie finansów. W niedługim czasie po Markowitzu zaczęto rozwijać tę ideę, czego skutkiem były między innymi opublikowana przez Jamesa Tobina w 1958 praca na temat \textit{super-efektywnego portfela}. Poźniej, w 1964 William Sharpe wydał pracę w której opracował \textit{model wyceny aktywów kapitałowych - (ang. CAPM - Capital Asset Pricing Model)}\cite{holton}\\

Zajmijmy się problemem analitycznego wyznaczenia portfela styczności. Poszukiwane są współrzędne portfela styczności (tzn. punktu $E$ na rysunku 3.5) oraz jego wagi. Jest to przypadek, w którym żadne środki nie są przeznaczone na aktywo wolne od ryzyka, a więc $w_f = 0$. Przekształćmy równania z podpunktu 3.2, aby znajdowały się w nich 3 aktywa: 

\begin{equation}
	w_1 + w_2 + w_f = 1,
\end{equation}
\begin{equation}
	\mu = w_1\mu_1 + w_2\mu_2 + w_fr_f,
\end{equation}
\begin{equation}
	\sigma^2 = w_1^2\sigma_1^2 + 2\sigma_{12}w_1w_2 + w_2^2\sigma_2^2,
\end{equation}
gdzie:\\
$\sigma_{12}$ - kowariancja zmiennych losowych $R_1$ oraz $R_2$.\\
Celem będzie znalezienie minimalnej wartości $\sigma$ przy warunku 3.25 i 3.26. Postępując analogicznie, jak w sekcji 3.2, na początek możemy pozbyć się jednej niewiadomej z równania opisującego wagi - niech będzie to $w_f$:

\begin{equation}
	w_f = 1 - w_1 - w_2,
\end{equation}
Podstawiając otrzymaną wagę $w_f$ do równania 3.26 można pozbyć się jednego równania, a więc celem będzie znalezienie minimalnej wartości $\sigma$ przy spełnieniu tylko jednego założenia - równania 3.29:

\begin{equation}
	\mu - r_f = (\mu_1 - r_f)w_1 + (\mu_2 - r_f)w_2.
\end{equation}
Aby znaleźć minimum funkcji $\sigma$, użyjemy mnożników Lagrange'a\cite{lagrange}. W tym celu, aby uzyskać funkcję ograniczającą w odpowiedniej dla mnożników Lagrange'a postaci $g(w_1, w_2) = 0$, przekształamy równanie 3.29 do następującej postaci:

\begin{equation}
g(w_1, w_2) = (\mu_1 - r_f)w_1 + (\mu_2 - r_f)w_2 - \mu + r_f = 0.
\end{equation}
Stąd:

\begin{equation}
	\nabla\\g = (\frac{\partial\\g}{w_1}, \frac{\partial\\g}{w_2}) = (\mu_1 - r_f, \mu_2 - r_f),
\end{equation}

\begin{equation}
	\nabla\sigma_p = (\frac{\partial\sigma}{w_1}, \frac{\partial\sigma}{w_2}) = (2{\sigma_1}^2w_1 + 2\sigma_{12}w_2, 2{\sigma_2}^2w_2 + 2\sigma_{12}w_1),
\end{equation}
Z metody mnożników Langrange'a otrzymujemy:

\begin{equation}
	\left( \begin{array}{cc}
		{\sigma_1}^2 & \sigma_{12}\\
		\sigma_{12} & {\sigma_2}^2
	\end{array} \right)
	\left( \begin{array}{c}
		z_1\\
		z_2
	\end{array} \right) =
	\left( \begin{array}{c}
		\mu_1 - r_f\\
		\mu_2 - r_f
	\end{array} \right),
\end{equation}
Pierwszym człon równania 3.33 to \textit{macierz wariancji-kowariancji}. Zawiera ona kowariancje wszystkich aktywów ryzykownych występujących w portfelu, a na jej przekątnej znajdują się wariancje tych aktywów. Nie może ona być określona dowolnie - jest ona zawsze kwadratowa, symetryczna, a jej wyznacznik jest nieujemny.
Zmienne $z_1$ oraz $z_2$ w równaniu 3.33 są zmiennymi pomocniczymi, które wyznaczają wagi $w_j$ portfela (równania 3.34 oraz 3.35):

\begin{equation}
	w_1 = \frac{z_1}{z_1 + z_2},
\end{equation}
\begin{equation}
	w_2 = \frac{z_2}{z_1 + z_2}.
\end{equation}
Współrzędne portfela styczności wynoszą zatem:

\begin{equation}
	\mu_T = w_1\mu_1 + w_2\mu_2,
\end{equation}

\begin{equation}
	\sigma_T = \sqrt{w_1^{2}\sigma_1^{2} + w_2^{2}\sigma_2^{2} + 2w_1w_2\sigma_{12}},
\end{equation}
za $w_1$ oraz $w_2$ podstawiając wartości wyliczone odpowiednio równaniami 3.34 i 3.35\cite{book}. Pamiętajmy, że wagi $w_1$ oraz $w_2$ są tutaj wagami aktywów ryzykownych - nie będą się więc one zmieniać w stosunku do siebie w zależności od inwestora.\\

Warto również wyznaczyć równanie prostej portfeli wydajnych, służy do tego równanie prostej przechodzącej przez dwa punkty:

\begin{equation}
	y = y_1 + \frac{y_2 - y_1}{x_2 - x_1}(x - x_1),
\end{equation}
do którego podstawiamy współrzędne punktu $(0, r_f)$ oraz otrzymane współrzędne portfela styczności $(\sigma_T, \mu_T)$:

\begin{equation}
	\mu = r_f + \frac{\mu_T - r_f}{\sigma_T}\sigma.
\end{equation}

\section{Portfel optymalny dla danego inwestora}
W rozważaniach przyjęto portfel składający się z dwóch aktywów ryzykownych oraz aktywa wolnego od ryzyka taki sam jak w podrozdziale 3.5. Chcąc w takim przypadku znaleźć portfel optymalny dla danego inwestora, należy znaleźć taki portfel wydajny (tzn. leżący na prostej portfeli wydajnych), który maksymalizuje funkcję użyteczności awersji do ryzyka. Można graficznie pokazać, że portfel optymalny dla naszego inwestora graficznie jest portfelem $O$ na rysunku 3.6:\\
\newpage
\begin{figure}[h]
	\centering
	\includegraphics[scale=0.4]{wykres_markowitz_ryzyk_z_u}
	\caption{Wykres paraboli funkcji użyteczności danego inwestora U z zaznaczonym punktem styczności $O$ do prostej opisującej portfele wydajne zawierające aktywo wolne od ryzyka.}
\end{figure}

Znając współczynnik awersji do ryzyka $k$ dla danego inwestora oraz mając wyznaczone równanie prostej portfeli wydajnych z równania 3.39, portfel znajdujemy, korzystając z wyznaczonych w podpunkcie 3.3 równań 3.21 i 3.22:

\begin{equation}
	\mu^* = \mu_0 + \frac{1}{2Ak},
\end{equation}

\begin{equation}
	\sigma^* = \sqrt{{\sigma_0}^2+\frac{1}{4Ak^2}}.
\end{equation}
W równaniach tych za $A$, $\sigma_0$ oraz $\mu_0$ podstawiamy wartości wyliczone odpowiednio za pomocą równań 3.10, 3.11 oraz 3.12, gdzie:
\begin{itemize}
	\item za wartości ($\mu_1$, $\sigma_1$) podstawiamy dane aktywa wolnego od ryzyka, a więc ($r_f$, $0$),
	\item za wartości ($\mu_2$, $\sigma_2$) podstawiamy dane portfela styczności, a więc ($\mu_T$, $\sigma_T$),
	\item korelacja $\rho = 1$.
\end{itemize} 
Otrzymany punkt o współrzędnych ($\mu^*, \sigma^*$) stanowi punkt portfela optymalnego danego inwestora. Wagi wyznaczamy również analogicznie, jak w podpunkcie 3.3 - przy czym trzeba pamiętać, że i one mają w tym przypadku inne znaczenie - są to wagi inwestycji w aktywo wolne od ryzyka oraz w portfel aktywów ryzykownych (traktowany jako jedno aktywo) - stąd otrzymujemy z równania 3.7:

\begin{equation}
w_1 = \frac{\mu^* - \mu_T}{r_f - \mu_T}.
\end{equation}

Zatem, przykładowo interpretacja wyniku przy budżecie 100 PLN oraz wagach $w_1 = 65\%, w_2 = 35\%$ brzmiałaby następująco: "Zainwestuj 65 PLN w aktywo wolne od ryzyka, a 35 PLN zainwestuj w aktywa ryzykowne w proporcjach wyznaczonych w równaniach 3.34 oraz 3.35".

\section{Aktualizacja portfela przy zmianie stopy wolnej od ryzyka}
W rozważaniach przyjęto portfel składający się z dwóch aktywów ryzykownych oraz aktywa wolnego od ryzyka taki sam jak w sekcji 3.5. Załóżmy, że chcemy w szybki sposób uaktualniać portfel styczności, gdy zmienia się stopa zwrotu z aktywa wolnego od ryzyka.
W takim przypadku można przekształcić równanie 3.33 przez przemnożenie obu jego stron przez macierz wariancji-kowariancji. Wynikiem tego działania jest:

\begin{equation}
	\left( \begin{array}{c}
		z_1\\
		z_2\\
	\end{array} \right) = 
	\left( \begin{array}{cc}
		c_{11}& c_{12}\\
		c_{21}& c_{22}\\
	\end{array} \right)
	\left( \begin{array}{c}
		1\\
		r_f\\
	\end{array} \right),
\end{equation}
gdzie macierz zawierająca wiersze $c_{nm}$ zawiera stałe pomocnicze, które zależą jedynie od aktywów ryzykownych. Sposób ten jest na tyle wygodny, że posiadając wartości macierzy zmiennych $z$ dla dwóch różnych wartości $r_f$, można z układu równań wyznaczyć macierz wierszy $c_{nm}$ i od tej pory szybko aktualizować wagi portfela w zależności od stopy zwrotu aktywa wolnego od ryzyka. Macierz wierszy $c_{nm}$ zawsze posiada dwie kolumny, a ilość wierszy odpowiada ilości aktywów ryzykownych w portfelu. W odróżnieniu od macierzy wariancji-kowariancji, ta macierz nie musi być symetryczna.

\section{Portfel styczności - dowolna liczba aktywów ryzykownych}
W tym punkcie w rozważaniach przyjęto $n$ aktywów ryzykownych w portfelu oraz jedno aktywo bez ryzyka o stopie zwrotu $r_f$. Zadaniem jest znalezienie portfela styczności, podobnie jak w sekcji 3.5. Wzory wyglądają więc podobnie, jednak tym razem posiadamy $n$ aktywów:

\begin{equation}
	\mu = w_1\mu_1 + w_2\mu_2 + ... + w_n\mu_n + w_fr_f,
\end{equation}

\begin{equation}
	w_1 + w_2 + \dots + w_n + w_f = 1.
\end{equation}
Podobnie jak poprzednio korzystając z metody mnożników Lagrange'a otrzymujemy:

\begin{equation}
	\left( \begin{array}{cccc}
		{\sigma_1}^2& \sigma_{12}  & \dots & \sigma_{1n}\\
		\sigma_{12} & {\sigma_2}^2 & \dots & \sigma_{2n}\\
		\vdots 		& \vdots 	   &       & \vdots  \\
		\sigma_{n1} & \sigma_{n2}  & \dots & \sigma_{n}^2\\
	\end{array} \right)
	\left( \begin{array}{c}
		z_1\\
		z_2\\
		\vdots\\
		z_n
	\end{array} \right) =
	\left( \begin{array}{c}
		\mu_1 - r_f\\
		\mu_2 - r_f\\
		\vdots\\
		\mu_n - r_f
	\end{array} \right),
\end{equation}
gdzie poszczególne wagi aktywów ryzykownych portfela styczności oblicza się analogicznie jak w podpunkcie 3.5:

\begin{equation}
w_j = \frac{z_j}{z_1 + \dots + z_n},
\end{equation}
gdzie $j = \{1, 2, 3, \dots, n\}$.\\
Można pokazać, że kolumnę zmiennych $z_j$ można przedstawić jako kolumnę ${\lambda}w_j$:

\begin{equation}
	\left( \begin{array}{cccc}
		{\sigma_1}^2& \sigma_{12}  & \dots & \sigma_{1n}\\
		\sigma_{12} & {\sigma_2}^2 & \dots & \sigma_{2n}\\
		\vdots 		& \vdots 	   &       & \vdots  \\
		\sigma_{n1} & \sigma_{n2}  & \dots & \sigma_{n}^2\\
	\end{array} \right)
	\left( \begin{array}{c}
		{\lambda}w_1\\
		{\lambda}w_2\\
		\vdots\\
		{\lambda}w_n
	\end{array} \right) =
	\left( \begin{array}{c}
		\mu_1 - r_f\\
		\mu_2 - r_f\\
		\vdots\\
		\mu_n - r_f
	\end{array} \right),
\end{equation}
gdzie:\\ 
$\lambda$ - stosunek zysku do zmienności,\\
$w_j$ - waga aktywa $j$.\\
Stosunek zysku do zmienności $\lambda$ można zapisać wzorem:

\begin{equation}
	\lambda = \frac{\mu - r_f}{{\sigma}^2},
\end{equation}
gdzie $\mu$ oraz $\sigma$ to współrzędne naszego portfela styczności.\\

W przypadku zmieniającej się stopy zwrotu z aktywa wolnego od ryzyka, podobnie jak w sekcji 3.7 możemy wyznaczyć macierz zmiennych pomocniczych $c_{nm}$, w tym przypadku posiadającej $n$ wierszy:

\begin{equation}
	\left( \begin{array}{c}
		z_1\\
		z_2\\
		\vdots\\
		z_n
	\end{array} \right) = 
	\left( \begin{array}{cc}
		c_{11}& c_{12}\\
		c_{21}& c_{22}\\
		\vdots& \vdots\\
		c_{n1}& c_{n2}\\
	\end{array} \right)
	\left( \begin{array}{c}
		1\\
		r_f\\
	\end{array} \right).
\end{equation}

\chapter{Założenia projektowe}

Program ma spełniać poniższe założenia projektowe:
\begin{itemize}
	\item opcja zapisu stanu portfela do pliku oraz jego wczytanie z pliku,
	\item umożliwienie działania programu w dwóch trybach:
	\begin{itemize}
		\item dane o aktywach znajdujących się w portfelu pobierane z internetu (za pomocą \textit{Yahoo Finance API}),
		\item dane o aktywach znajdujących się w portfelu pobierane z pliku tekstowego,
	\end{itemize}
	\item możliwość ustalenia budżetu użytkownika,
	\item wyznaczanie portfela styczności,
	\item możliwość zapisu współczynnika awersji do ryzyka użytkownika,
	\item wyznaczanie portfela optymalnego dla użytkownika,
	\item możliwość określenia horyzontu czasowego, z jakiego mają zostać pobrane wyniki,
	\item możliwość wyboru, z ilu dni mają być liczone zwroty.
\end{itemize}

\chapter{Prezentacja oraz opis aplikacji}

Na podstawie założeń projektowych powstało narzędzie wyznaczające optymalny portfel inwestora, jego nazwa to \code{the-wallet}. Pełen kod programu został umieszczony w dodatku A tej pracy. Narzędzie zostało napisane w języku Python, w zgodności z wersją 3.7.8. Składa się ono z trzech plików: \code{main.py}, \code{functions.py} oraz \code{additional\_functions.py}. W tym rozdziale zostanie zaprezentowane działanie programu, a następnie omówiona funkcjonalność kodu.

\section{Przykładowe uruchomienie programu w trybie API}

Program - zgodnie z założeniami projektowymi - można uruchomić w dwóch trybach, różniących się sposobem pobierania danych o spółkach. W tym podrozdziale zostanie omówiony tryb pobierania danych przez API (dalej nazywanego w skrócie trybem API). Pobierane dane są danymi rzeczywistymi pochodzącymi z bazy \textit{Yahoo Finance} (https://finance.yahoo.com/). Aby dokonać uruchomienia programu w trybie API znajdując się w katalogu roboczym narzędzia należy wywołać komendę:
\mint{bash}{python3 main.py}, po udanym uruchomieniu programu ukazuje on użytkownikowi główne menu, tak jak na rysunku 5.1.

\begin{figure}[ht]
	\centering
	\includegraphics[scale=0.44]{interfejs1}
	\caption{Główne menu programu.}
\end{figure}
Nawigacja po opcjach odbywa się poprzez wpisanie z klawiatury numeru interesującej nas opcji. 
W tym momencie nie posiadamy żadnego istniejącego portfela, więc chcemy rozpocząć pracę na nowym portfelu. Wybieramy z klawiatury opcję numer 2, która powoduje przejście do kolejnego menu - menu portfela (zostało to uwiecznione na rysunku 5.2).
\newpage
\begin{figure}[ht]
	\centering
	\includegraphics[scale=0.44]{interfejs2}
	\caption{Menu portfela.}
\end{figure}
Nasz portfel jest w tym momencie pusty - nie posiadamy żadnych aktywów ryzykownych ani aktywa wolnego od ryzyka. Dodajemy dwa aktywa ryzykowne - niech będą to aktywa Apple'a oraz Facebooka. Aby tego dokonać, po wybraniu opcji numer 1 należy podać ich symbole, pod którymi aktywa te są identyfikowane na giełdzie. Po sprawdzeniu na stronie \textit{https://finance.yahoo.com/} możemy zobaczyć, że symbolami tych aktyw są odpowiednio kody AAPL i FB. Proces dodania aktywów do portfela widoczny jest na rysunkach 5.3a oraz 5.3b.

\begin{figure}[hb]
	\centering
	\subfloat[Dodanie do portfela aktyw firmy Apple.]{\label{aapl}
	\includegraphics[width=0.4\textwidth]{apple}}
	\quad
	\subfloat[Dodanie do portfela aktyw firmy Facebook.]{\label{fb}
	\includegraphics[width=0.4\textwidth]{fb}}
	\caption{Dodanie aktyw ryzykownych do portfela.}
\end{figure}

Następny krok w tworzeniu w pełni funkcjonalnego portfela to dodanie aktywa wolnego od ryzyka - w tym celu należy wybrać opcję 3 i podać roczną stopę zwrotu z takiego aktywa - podajmy wartość $2\%$.
\newpage
\begin{figure}[ht]
	\centering
	\includegraphics[scale=0.4]{rfassetapi}
	\caption{Dodanie do portfela aktywa wolnego od ryzyka.}
\end{figure}

Dla poprawnego działania narzędzia portfel musi posiadać zdefiniowaną funkcję użyteczności awersji od ryzyka. W tym celu po wybraniu opcji 4 podajemy współczynnik $k$ awersji do ryzyka inwestora - niech wynosi on $k=0,1$.

\begin{figure}[ht]
	\centering
	\includegraphics[scale=0.4]{kapi}
	\caption{Dodanie do portfela współczynnika $k$ awersji do ryzyka inwestora.}
\end{figure}

Przez wybór opcji 5, 6 oraz 7 użytkownik może dostosować odpowiednio:
\begin{itemize}
	\item zakres dni z których rozpatrujemy dane,
	\item sposób liczenia średniej ceny zamknięcia (średnia jednodniowa/tygodniowa/miesięczna/trzymiesięczna),
	\item swój budżet, dzięki któremu otrzymane wagi będą przeliczane na odpowiednią sumę pieniędzy które należy zainwestować w dane aktywo.
\end{itemize}
Parametry te w momencie tworzenia obiektu portfela mają swoje następujące wartości domyślne:
\begin{itemize}
	\item zakres dni z których rozpatrujemy dane wynosi 500 dni (w trybie API),
	\item domyślna średnia cena aktywów jest średnią jednodniową,
	\item budżet użytkownika wynosi 100 USD.
\end{itemize}
Na potrzeby obecnego uruchomienia nie będziemy zmieniać tych wartości. Sprawdźmy jak wygląda w tej chwili portfel. Umożliwia to kryjąca się pod numerem 0 opcja \code{Show wallet}. Gotowy portfel ukazano na rysunku 5.6.
\newpage
\begin{figure}[ht]
	\centering
	\includegraphics[scale=0.3]{showwalletapi}
	\caption{Gotowy portfel w trybie API.}
\end{figure}

Zobaczmy, jak wygląda portfel optymalny dla wprowadzonych danych - w tym celu używamy opcji numer 8 - \code{Show optimal portfolio weights}. Wynik w tym momencie nie będzie podlegał interpretacji (na interpretacje przeznaczony jest rozdział szósty tej pracy).

\begin{figure}[ht]
	\centering
	\includegraphics[scale=0.4]{optimalportfolioapi}
	\caption{Przykładowy optymalny portfel w trybie API.}
\end{figure}


\section{Przykładowe uruchomienie programu w trybie file-mode}

Uruchomienie w trybie pobierania danych z pliku tekstowego o cenach zamknięcia spółek (dalej nazywanego w skrócie \textit{file-mode}) wymaga od nas odpowiedniego przygotowania takowego pliku. Przykład odpowiednio sformatowania został przedstawiony w poniższym bloku kodu:
\newpage
\begin{minted}{bash}
	Stock1;Stock2;Stock3
	2.22;2.34;3.71
	2.22;2.35;3.78
	2.24;2.32;3.79
	2.19;2.37;3.81
	2.24;2.38;3.81
	2.25;2.40;3.82
	2.26;2.41;3.84
	2.22;2.40;3.87
\end{minted}
W pierwszym wierszu należy podać oddzielone średnikami nazwy spółek, których kursy zamknięcia będą rozpatrywane w narzędziu. Następnie, w kolejnych wierszach należy podawać chronologicznie od najstarszego do najnowszego dzienne kursy zamknięcia dla spółek, tworząc w ten sposób swego rodzaju tabelę. Komenda uruchomienia programu w opisywanym trybie wygląda następująco:
\mint{bash}{python3 main.py -f dane.txt},
gdzie dane.txt jest ścieżką do pliku z przygotowanymi danymi. Oczywiście nazwa \code{dane.txt} jest przykładowa - plik może nazywać się dowolnie i mieć dowolne rozszerzenie.
Na potrzeby przykładu przygotowano dane, które zapisano do pliku tekstowego o nazwie \code{mydata.txt}. Jego zawartość ukazano na rysunku 5.8. 

\begin{figure}[ht!]
	\centering
	\includegraphics[scale=0.35]{catmydata}
	\caption{Przygotowany plik tekstowy \code{mydata.txt} do uruchomienia w trybie \textit{file-mode}.}
\end{figure}

Uruchomienie programu zostało przedstawione na rysunku 5.9. Zostajemy przywitani komunikatem o tym, że narzędzie działa w trybie \textit{file-mode} i pomijane jest główne menu - od razu zostajemy przeniesieni do menu portfela.

\begin{figure}[ht!]
	\centering
	\includegraphics[scale=0.4]{interfejs2filemode}
	\caption{Uruchomienie narzędzia w trybie \textit{file-mode}.}
\end{figure}


Dane do portfela wprowadzamy dokładnie w taki sam sposób jak w trybie API. Na potrzeby tego przykładu ustalamy poziom rocznego zwrotu z aktywa wolnego od ryzyka na poziomie $2 \%$, a współczynnika $k$ awersji do ryzyka inwestora na $k=0,7$. Dla reszty parametrów pozostawiamy ich domyślne wartości. Gotowy portfel został przedstawiony na rysunku 5.10.

\begin{figure}[ht]
	\centering
	\includegraphics[scale=0.37]{showwalletfilemode}
	\caption{Gotowy portfel w trybie \textit{file-mode}.}
\end{figure}
Jak możemy zauważyć, program dostosował zakres dni z których rozpatrujemy dane do długości wprowadzonych danych - wprowadziliśmy 30 cen, więc domyślną wartością zakresu jest teraz 30 dni.
Sprawdźmy, jak wygląda portfel optymalny dla wprowadzonych danych. Wynik znajduje się na rysunku 5.11.
\newpage
\begin{figure}[ht!]
	\centering
	\includegraphics[scale=0.4]{optimalportfoliofilemode}
	\caption{Przykładowy optymalny portfel w trybie \textit{file-mode}.}
\end{figure}

Portfel na którym wykonujemy operacje możemy zapisać do pliku, aby móc z niego korzystać przy kolejnych uruchomieniach programu. W tym celu wybieramy opcję \code{Save wallet to file} a następnie wprowadzamy ścieżkę do pliku do którego chcemy zapisać nasz portfel. Możliwe jest nadpisanie istniejącego już pliku, jednak w takim przypadku program przed operacją wyświetla stosowny komunikat i prosi o dodatkowe potwierdzenie, czy na pewno chcemy nadpisać plik. Zapisany portfel odczytujemy, wybierając w głównym menu przy kolejnym uruchomieniu programu opcję \textit{Open existing wallet}. Cały opisany proces został przedstawiony na rysunku 5.12.

\begin{figure}[ht!]
	\centering
	\includegraphics[scale=0.342]{openwlt}
	\caption{Zapis portfela do pliku oraz odczytanie portfela.}
\end{figure}
\newpage
\section{main.py - opis funkcjonalności}
Po zaprezentowaniu przykładowego działania programu pokażemy jak wygląda działanie kodu programu. Dla łatwiejszego zrozumienia przygotowano diagramy sekwencyjne, które będą umieszczane po pokazaniu kodu danej sekwencji.
Zacznijmy od pliku \code{main.py}. Jest to plik za pomocą którego uruchamiany jest program, zawiera on import zewnętrznej biblioteki \code{argparse}, wewnętrznej \code{functions} oraz jedną funkcję - \code{main} - przedstawioną na rysunku 5.13.

\begin{figure}[ht]
	\centering
	\includegraphics[scale=0.37]{main}
	\caption{Funkcja \code{main}.}
\end{figure}

Funkcja ta ma dwa zadania: przetworzenie opcjonalnego argumentu \code{-f} do uruchomienia trybu \textit{file-mode} oraz uruchomienie interfejsu programu, znajdującego się w pliku \code{functions.py}.\\

\begin{figure}[ht]
	\centering
	\includegraphics[scale=0.65]{main_schema}
	\caption{Schemat działania kodu w pliku \code{main.py}.}
\end{figure}

\section{functions.py oraz additional\_functions.py - opis funkcjonalności}

Plik \code{functions.py} jest trzonem działania programu - znajdują się z nim dwie główne klasy: \code{Wallet} i \code{Interface} oraz jedna pomocnicza \code{Stock}.
Klasa pomocnicza \code{Stock} jest używana tylko podczas uruchomionego trybu \textit{file-mode} i symuluje ona strukturę danych pobranych za pomocą \textit{Yahoo Finance API}, dzięki czemu możliwe jest jednakowy sposób działania funkcji kalkulacyjnych programu dla obu trybów działania (poza funkcją \code{get\_stocks\_returns}).
\begin{figure}[ht]
	\centering
	\includegraphics[scale=0.37]{classstock}
	\caption{Klasa \code{Stock}.}
\end{figure}

Klasę \code{Wallet} ukazano na rysunku 5.16. Przechowuje ona zmienne związane bezpośrednio z portfelem oraz wyniki kalkulacji wykonywanych w celu obliczenia portfela optymalnego dla inwestora. Oprócz tego, posiada swoją definicję funkcji \code{\_\_str\_\_}, dzięki której użycie w programie formuły \code{print} na obiekcie klasy \code{Wallet} zwraca wszystkie główne informacje dotyczące obecnego stanu portfela (wynik uruchomienia możemy zobaczyć na rysunkach 5.6 oraz 5.10).

\begin{figure}[ht]
	\centering
	\includegraphics[scale=0.37]{classwallet}
	\caption{Klasa \code{Wallet}.}
\end{figure}

Kolejna składowa pliku to klasa \code{Interface} - jej obiekt jest tworzony w pliku \code{main.py}. Funkcja inicjalizacyjna obiektu tej klasy najpierw sprawdza, w jakim trybie zostało uruchomione narzędzie, a następnie w zależności od trybu ma różne działanie:
\begin{itemize}
	\item \textbf{API} - wyświetla główne menu, a następnie po wybraniu odpowiedniej opcji wyświetla menu portfela dla trybu API,
	\item \textbf{text-mode} - ładuje dane z pliku za pomocą funkcji \code{load\_stocks\_info\_from\_file} a następnie wyświetla menu portfela dla trybu \textit{file-mode}.
\end{itemize}

\begin{figure}[ht]
	\centering
	\includegraphics[scale=0.4]{initinterface}
	\caption{Funkcja inicjalizacyjna obiektu klasy \code{Interface}.}
\end{figure}
\newpage
Funkcja \code{load\_stocks\_info\_from\_file} (ukazana na rysunku 5.18) w pierwszej kolejności tworzy obiekt klasy \code{Wallet} w obiekcie klasy \code{Interface} - jest to portfel na którym będziemy operować w tym uruchomieniu programu. Następnie funkcja zamienia strumień wejściowy (czyli dane z naszego pliku) na strukturę \code{DataFrame} z biblioteki \code{pandas}. Następuje sprawdzenie, czy plik jest odpowiednio sformatowany oraz czy oprócz nazw spółek i separatorów \code{;} zawiera jedynie dane numeryczne (jeżeli tak nie jest, działanie programu zostaje przerwane) - poźniej dane są zapisywane do listy aktywów ryzykownych \code{stocks} w portfelu jako obiekty klasy pomocnicznej \code{Stock}.

\begin{figure}[ht]
	\centering
	\includegraphics[scale=0.37]{loadfromfilecode}
	\caption{Funkcja \code{load\_stocks\_info\_from\_file} klasy \code{Interface}.}
\end{figure}

\begin{figure}[ht]
	\centering
	\includegraphics[scale=0.5]{initinterface_schema}
	\caption{Schemat działania funkcji \code{\_\_init\_\_} klasy \code{Interface}.}
\end{figure}
\newpage
Funkcja \code{main\_menu} klasy \code{Interface} pozwala na wybór jednej z trzech opcji: odczytania portfelu z pliku, otwarcia nowego portfela (tzn. utworzenie obiektu klasy \code{Wallet} w obiekcie klasy \code{Interface}) lub opuszczenia programu. Menu jest zabezpieczone przed przypadkiem wpisania w strumieniu wejściowym czegokolwiek poza dozwolonymi symbolami (czyli na przykład w tym przypadku cyfry \code{1}, \code{2} lub \code{3}) przez komunikat o nieprawidłowo wybranej funkcji i powrót do możliwości wpisania prawidłowego symbolu.

\begin{figure}[ht!]
	\centering
	\includegraphics[scale=0.37]{mainmenucode}
	\caption{Funkcja \code{main\_menu} klasy \code{Interface}.}
\end{figure}
\newpage
Po wybraniu opcji otworzenia portfela z pliku uruchamiana jest funkcja \code{open\_existing} \code{\_wallet}. W pierwszej kolejności prosi nas o wpisanie ścieżki do pliku, gdzie znajduje się portfel na którym chcemy działać w tym uruchomieniu programu. Jeżeli zarówno ścieżka jak i portfel istnieją, obiekt portfela jest tworzony z obiektu zapisanego do pliku za pomocą biblioteki \code{pickle}. Jeżeli otwierany przez nas portfel był tworzony w trybie \textit{file-mode}, wyświetlany jest o tym odpowiedni komunikat. Funkcja działa w pętli do momentu wpisania odpowiedniej ścieżki zawierającej prawidłowo zapisany portfel.

\begin{figure}[ht!]
	\centering
	\includegraphics[scale=0.37]{openfromfilecode}
	\caption{Funkcja \code{open\_existing\_wallet} klasy \code{Interface}.}
\end{figure}

\begin{figure}[ht!]
	\centering
	\includegraphics[scale=0.54]{mainmenu_schema}
	\caption{Schemat działania funkcji \code{main\_menu} klasy \code{Interface}.}
\end{figure}
\newpage
Przejdźmy do opisu menu portfela, które posiada dwie wersje - \code{wallet\_menu\_api} oraz \newline
\code{wallet\_menu\_file\_mode}. W omawianiu funkcji skupimy się na tym pierwszym, gdyż jest to po prostu wzbogacona wersja menu portfela w trybie \textit{file-mode}, a działanie dostępnych funkcji w obu trybach jest takie samo dzięki pomocniczej klasie \code{Stock}.
Funkcja \code{wallet\_menu\_api} posiada do wyboru jedenaście opcji, wybieranych przez wpisanie z klawiatury opowiedniej cyfry od \code{0} do \code{10}. Podobnie jak \code{main\_menu}, obecne jest zabezpieczenie przed przypadkiem wpisania czegokolwiek poza dozwolonymi symbolami. 

\begin{figure}[ht!]
	\centering
	\includegraphics[scale=0.35]{walletmenucode}
	\caption{Funkcja \code{wallet\_menu\_api} klasy \code{Interface}.}
\end{figure}

\begin{figure}[ht!]
	\centering
	\includegraphics[scale=0.43]{walletmenu_schema}
	\caption{Schemat działania funkcji \code{wallet\_menu\_api} klasy \code{Interface}.}
\end{figure}
\newpage
\textcolor{white}{.}
\newpage
Funkcje \code{add\_stock\_to\_wallet} oraz \code{remove\_stock\_from\_wallet} są dostępne jedynie w trybie \textit{API}. Umożliwiają modyfikację listy posiadanych aktywów ryzykownych w portfelu. Funkcja dodająca aktywo do portfela przyjmuje od użytkownika symbol aktywa, pod którym jest ono dostępne w bazie \textit{Yahoo Finance}. Następnie sprawdza, czy symbol ten istnieje w bazie - jeżeli tak, aktywo jest dodawane do portfela. W przeciwnym przypadku zwracany jest komunikat o błędzie.
Funkcja usuwająca aktywo z portfela również przyjmuje od użytkownika symbol aktywa, którego chce się pozbyć. Obecność aktywa w portfelu jest weryfikowana i w przypadku istnienia aktywa w portfelu jest ono usuwane.

\begin{figure}[ht!]
	\centering
	\includegraphics[scale=0.37]{addremovestockcode}
	\caption{Funkcje \code{add\_stock\_to\_wallet} oraz \code{remove\_stock\_from\_wallet} klasy \code{Interface}.}
\end{figure}

\begin{figure}[ht!]
	\centering
	\includegraphics[scale=0.5]{addstock_schema}
	\caption{Schemat działania funkcji \code{add\_stock\_to\_wallet} klasy \code{Interface}.}
\end{figure}

\begin{figure}[ht!]
	\centering
	\includegraphics[scale=0.5]{removestock_schema}
	\caption{Schemat działania funkcji \code{remove\_stock\_from\_wallet} klasy \code{Interface}.}
\end{figure}
\newpage
Opcja pod numerem \code{3} uruchamia funkcję \code{update\_risk\_factor}, aktualizująca współczynnik awersji do ryzyka inwestora. Program sprawdza czy wprowadzona wartość jest większa niż $0$ oraz upewnia się że jest ona możliwa do zapisu jako zmienna typu \textit{float} (liczba zmiennoprzecinkowa).

\begin{figure}[ht]
	\centering
	\includegraphics[scale=0.37]{updatekcode}
	\caption{Funkcja \code{update\_risk\_factor} klasy \code{Interface}.}
\end{figure}

\begin{figure}[ht!]
	\centering
	\includegraphics[scale=0.5]{updatek_schema}
	\caption{Schema działania funkcji \code{update\_risk\_factor} klasy \code{Interface}.}
\end{figure}
\newpage
Funkcja \code{update\_risk\_free\_asset} przyjmuje roczny zwrot z aktywa wolnego od ryzyka w postaci procentowej, po czym zamienia go na liczbę. Wpisywana wartość musi znajdować się w przedziale od 0 do 100, w innym przypadku zwracany jest komunikat z błędem. Funkcja działa w pętli do momentu wpisania wartości spełniającej te warunki.

\begin{figure}[ht]
	\centering
	\includegraphics[scale=0.37]{updaterfcode}
	\caption{Funkcja \code{update\_risk\_free\_asset} klasy \code{Interface}.}
\end{figure}

\begin{figure}[ht]
	\centering
	\includegraphics[scale=0.5]{updaterf_schema}
	\caption{Schemat działania funkcji \code{update\_risk\_free\_asset} klasy \code{Interface}.}
\end{figure}
\newpage
Funkcja \code{update\_amount\_of\_days\_to\_take\_data\_from} aktualizuje zakres dni z których rozpatrujemy dane dotyczące aktywów ryzykownych. Domyślna wartość wynosi 500 dni w trybie API oraz $n$ dni dla danych z $n$ dni podanych w trybie \textit{file-mode}. Funkcja przyjmuje od użytkownika ilość dni, a więc liczbę całkowitą. Liczba wpisana przez użytkownika musi:
\begin{itemize}
	\item być nieujemna,
	\item być mniejsza niż zakres dostępnych danych, co jest sprawdzane przez funkcję \code{given\_da} \code{ta\_is\_long\_enough} importowaną z pliku \code{additional\_functions.py} (warunek sprawdzany tylko w trybie \textit{file-mode}).
\end{itemize}
Funkcja działa w pętli do momentu wpisania wartości spełniającej te warunki.

\begin{figure}[ht]
	\centering
	\includegraphics[scale=0.37]{updatedayscode}
	\caption{Funkcja \code{update\_amount\_of\_days\_to\_take\_data\_from} klasy \code{Interface}.}
\end{figure}

\begin{figure}[ht!]
	\centering
	\includegraphics[scale=0.37]{givendatalongenoughcode}
	\caption{Funkcja \code{given\_data\_is\_long\_enough}.}
\end{figure}

\begin{figure}[ht]
	\centering
	\includegraphics[scale=0.5]{updatedays_schema}
	\caption{Schemat działania funkcji \code{update\_amount\_of\_days\_to\_take\_data\_from} klasy \code{Interface}.}
\end{figure}
\newpage
Funkcja \code{update\_return\_calculation} aktualizuje ilość dni z których liczona jest średnia cena zamknięcia aktywów ryzykownych. Domyślna wartość tej zmiennej wynosi 1 dzień. Użytkownik ma do wyboru cztery opcje:
\begin{itemize}
	\item średnie jednodniowe (\code{1d}),
	\item średnie tygodniowe (\code{1wk}),
	\item średnie miesięczne (\code{1mo}),
	\item średnie trzymiesięczne (\code{3mo}).
\end{itemize}
Funkcja działa do momentu wprowadzenia jednej z tych wartości - sama wartość typu string jest potrzebna w funkcji \code{get\_stocks\_returns}, jako jeden z argumentów przy pobieraniu danych, natomiast wartość liczbowa potrzebna do obliczeń jest uzyskiwana dzięki mapowaniu wartości typu string na int przez zmienną globalną \code{N\_DAYS\_RETURNS\_MAP}.
Oprócz tego, wybrana wartość musi być mniejsza niż zakres dostępnych danych, co jest sprawdzane przez funkcję \code{given\_data\_is\_long\_enough} importowaną z pliku \code{additional} \code{\_functions.py}(warunek sprawdzany tylko w trybie \textit{file-mode}) oraz być mniejsza lub równa zakresowi dni z których rozpatrujemy dane dotyczące aktywów ryzykownych.
\begin{figure}[ht]
	\centering
	\includegraphics[scale=0.37]{updateretdayscode}
	\caption{Funkcja \code{update\_return\_calculation} klasy \code{Interface}.}
\end{figure}

\begin{figure}[ht!]
	\centering
	\includegraphics[scale=0.37]{ndaysretmap}
	\caption{Zmienna globalna \code{N\_DAYS\_RETURNS\_MAP}.}
\end{figure}

\begin{figure}[ht!]
	\centering
	\includegraphics[scale=0.55]{updateretdays_schema}
	\caption{Schemat działania funkcji \code{update\_return\_calculation} klasy \code{Interface}.}
\end{figure}
\newpage
Funkcja \code{update\_budget} aktualizuje budżet, dla którego będą liczone kwoty inwestycji w dane aktywa. Domyślna wartość to 100 USD. Wpisana przez użytkownika wartość musi być większa niż zero i być liczbą całkowitą. Funkcja działa do momentu wpisania wartości spełniającej te warunki.
\begin{figure}[ht!]
	\centering
	\includegraphics[scale=0.37]{updatebudgetcode}
	\caption{Funkcja \code{update\_budget} klasy \code{Interface}.}
\end{figure}

\begin{figure}[ht!]
	\centering
	\includegraphics[scale=0.55]{updatebudget_schema}
	\caption{Schemat działania funkcji \code{update\_budget} klasy \code{Interface}.}
\end{figure}

\newpage
Najbardziej rozbudowanym elementem programu jest funkcja \code{show\_optimal\_portfol} \code{io\_weights}. Na początek przelicza ona wprowadzony roczny zwrot z aktywa wolnego od ryzyka na okres, z jakiego liczony jest średni zwrot z aktywa ryzykownego (tzn. dla średnich jednodniowych wprowadzony roczny zwrot z aktywa wolnego od ryzyka jest przeliczany na dzienny zwrot). Następnie wyliczany jest portfel styczności, a więc wagi aktywów ryzykownych w portfelu. Są one wyświetlane za pomocą funkcji \code{show\_risk\_assets\_weights}. W następnej kolejności wyświetlana jest macierz kowariancji (jako jedyna jest wyświetlana zwykłą komendą \code{print} dla celów estetyczniejszego sposobu ukazania macierzy), oraz liczone są wagi portfela optymalnego. Ostatnim elementem funkcji jest wyświetlenie użytkownikowi zgodnie z wprowadzonym przez niego budżetem, jakie kwoty powinien on zainwestować w dane aktywa.

\begin{figure}[ht!]
	\centering
	\includegraphics[scale=0.37]{showoptimalweightscode}
	\caption{Funkcja \code{show\_optimal\_portfolio\_weights} klasy \code{Interface}.}
\end{figure}

\begin{figure}[ht!]
	\centering
	\includegraphics[scale=0.37]{showriskweightscode}
	\caption{Funkcja \code{show\_risk\_assets\_weights} klasy \code{Interface}.}
\end{figure}

\begin{figure}[ht!]
	\centering
	\includegraphics[scale=0.37]{showcovmartixcode}
	\caption{Funkcja \code{show\_covariation\_matrix} klasy \code{Interface}.}
\end{figure}
\newpage
Funkcja \code{calculate\_tangent\_portfolio\_weights} na początku oblicza zwroty z aktywów ryzykownych znajdujących w naszym portfelu. Jeżeli wybrany tryb działania to \textit{API}, za pomocą funkcji \code{history} obiektu klasy \code{Ticker} z biblioteki \code{yfinance} pobierane są dane aktywa ryzykownego i zapisywane w obiekcie klasy \code{DataFrame}. Dla trybu \textit{file-mode} dane pobierane są z listy \textit{self.wallet.stocks} która została stworzona podczas działania funkcji \code{load\_stocks\_info\_from\_file}  Zwrot liczony jest przez obliczenie zmian procentowych wartości z kolumny \code{['Adj Close']}, czyli ceny zamknięcia z uwzględnionymi przypadkami dywidendy oraz podziału akcji - drugie z tych obliczeń jest dokonywane przez funkcji \code{apply\_stock\_splits} importowaną z pliku \code{additional\_functions.py}. Zwroty te zapisywane są w obiekcie klasy \code{DataFrame}.

\begin{figure}[ht!]
	\centering
	\includegraphics[scale=0.37]{calculatetangentcode}
	\caption{Funkcja \code{calculate\_tangent\_portfolio\_weights} klasy \code{Interface}.}
\end{figure}

\begin{figure}[ht!]
	\centering
	\includegraphics[scale=0.37]{getstocksreturnscode}
	\caption{Funkcja \code{get\_stocks\_returns} klasy \code{Interface}.}
\end{figure}
\newpage
Następnie z pomocą funkcjonalności \textit{list comprehension} języka \textit{Python}\cite{pythonlist} oraz funkcji \code{calculate\_z\_matrix} obliczane są zmienne pomocnicze \textit{z} (ich rola jest wyjaśniona w podrozdziale 3.5). Są one przeliczane na wagi aktywów ryzykownych w portfelu również za pomocą \textit{list comprehension}.

\begin{figure}[ht!]
	\centering
	\includegraphics[scale=0.37]{calczmatrixcode}
	\caption{Funkcja \code{calculate\_z\_matrix} klasy \code{Interface}.}
\end{figure}

Przejdźmy do omówienia funkcji \code{calculate\_optimal\_portfolio\_weights}. Na początku oblicza ona parametry portfela styczności - to znaczy jego oczekiwanego zwrotu i odchylenia standardowego. Nastepnie obliczane są parametry portfela optymalnego - oczekiwany zwrot, odchylenie standardowe oraz \textit{A} (tak jak zostało to wyjaśnione w podrozdziale 3.6). Wyliczone parametry pozwalają nam osiągnąć ostateczny cel programu - obliczenie wagi aktywa ryzykownego oraz portfela aktywów ryzykownych. Sformatowane do przyjaznego użytkownikowi formatu wyniki są przedstawiane za pomocą funkcji \code{show\_budget\_calculations}. Po zakończeniu działania funkcji program wraca do menu portfela.

\begin{figure}[ht!]
	\centering
	\includegraphics[scale=0.34]{calcoptportwcode}
	\caption{Funkcja \code{calculate\_optimal\_portfolio\_weights} klasy \code{Interface}.}
\end{figure}

\begin{figure}[ht!]
	\centering
	\includegraphics[scale=0.3]{gettanportparamscode}
	\caption{Funkcja \code{get\_tangent\_portfolio\_parameters} klasy \code{Interface}.}
\end{figure}

\begin{figure}[ht!]
	\centering
	\includegraphics[scale=0.3]{getoptportparamscode}
	\caption{Funkcja \code{get\_optimal\_portfolio\_parameters} klasy \code{Interface}.}
\end{figure}

\begin{figure}[ht!]
	\centering
	\includegraphics[scale=0.3]{showbudgetcalccode}
	\caption{Funkcja \code{show\_budget\_calculations} klasy \code{Interface}.}
\end{figure}
\newpage
\begin{figure}[ht!]
	\centering
	\includegraphics[scale=0.37]{showoptimalweights_schema}
	\caption{Schemat działania funkcji \code{show\_optimal\_portfolio\_weights} klasy \code{Interface}.}
\end{figure}
\newpage
\textcolor{white}{.}
\newpage
Funkcja \code{save\_wallet\_to\_file} umożliwia zapis aktualnie otwartego portfela do pliku. W tym celu użytkownik wpisuje ścieżkę, pod którą chciałby zapisać portfel. Jeżeli plik pod wskazaną ścieżką już istnieje, użytkownik stawiany jest przed wyborem nadpisania istniejącego już pliku aktualnym portfelem albo wybrania innej ścieżki. Zapis odbywa się za pomocą funkcji \code{dump} z biblioteki \code{pickle}. Funkcja działa do momentu poprawnego zapisu portfela.

\begin{figure}[ht]
	\centering
	\includegraphics[scale=0.3]{savewalletcode}
	\caption{Funkcja \code{save\_wallet\_to\_file} klasy \code{Interface}.}
\end{figure}

\begin{figure}[ht]
	\centering
	\includegraphics[scale=0.55]{savewallet_schema}
	\caption{Schemat działania funkcji \code{save\_wallet\_to\_file} klasy \code{Interface}.}
\end{figure}

\chapter{Ocena działania programu}

W tym rozdziale stworzony zostanie portfel zawierający aktywa ryzykowne oraz jedno aktywo wolne od ryzyka. Nie jest jasno określone ile aktywów ryzykownych powinno się posiadać w portfelu dla efektywnej dywersyfikacji. Istnieją artykuły naukowe o tym że liczba ta powinna przekraczać 50, jednak takie portfele są kosztowne i trudne do utrzymania przez przeciętnego inwestora, gdyż każda operacja kupna oraz sprzedaży wiąże się z pewnym kosztem. Z drugiej strony, znajdziemy artykuły o tym, że znaczne profity z dywersyfikacji możemy osiągać już przy 6-15 aktywach (przykłady:Ron Bird oraz Mark Tippet w swojej pracy \textit{Note---Naive Diversification and Portfolio Risk---A Note} proponowali liczbę 10-15 aktywów \cite{birdtippett}, natomiast Simone Brands oraz David R. Gallagher - 6 aktywów - jednak warto dodać że badanie wykonywano jedynie biorąc pod uwagę rynek australijski\cite{brandsgallagher})\cite{howmanystocks}. Stąd przystaniemy przy liczbie dziesięciu aktywów ryzykownych w portfelu.

Za pomocą zmiany danych w portfelu takich jak:

\begin{itemize}
	\item współczynnik awersji do ryzyka inwestora,
	\item średnia cena zamknięcie liczona z \textit{n} dni,
	\item ewentualna zmiana aktywa ryzykownego,
\end{itemize}
stworzymy kilka wersji portfela - wyniki zostaną przedstawione w tabelach oraz omówione. Portfele zostaną stworzone w programie działającym w trybie \textit{API}.
Na początek pokrótce przedstawmy aktywa ryzykowne, które będą znajdować się w portfelu - w nawiasach znajdują się symbole, pod którymi można znaleźć te aktywa na stronie \textit{Yahoo Finance} i w takiej formie aktywa te będą przedstawiane w tabelach:

\begin{itemize}
	\item \textit{\textbf{Tesla, Inc. (TSLA)}} - amerykański producent elektrycznych samochodów osobowych, SUV-ów, pickupów, samochodów sportowych oraz ciągników siodłowych\cite{tesla},
	\item \textit{\textbf{British American Tobacco p.l.c. (BTI)}} - brytyjskie przedsiębiorstwo zajmujące się wyrobem oraz sprzedażą produktów tytoniowych,
	\item \textit{\textbf{Facebook, Inc. (FB)}} - amerykański konglomerat technologiczny z siedzibą w Menlo Park w Kalifornii\cite{facebook},
	\item \textit{\textbf{Netflix, Inc. (NFLX)}} - telewizja internetowa, oferująca za zryczałtowaną opłatą dostęp do filmów i seriali poprzez media strumieniowe\cite{netflix},
	\item \textit{\textbf{Tilray, Inc. (TLRY)}} - kanadyjska firma farmaceutyczna i konopna\cite{tilray},
	\item \textit{\textbf{Adidas AG (ADDYY)}} - niemieckie przedsiębiorstwo produkujące obuwie i odzież sportową\cite{adidas},
	\item \textit{\textbf{Nokia Corporation (NOK)}} - fińskie przedsiębiorstwo zajmujące się technologiami telekomunikacyjnymi\cite{nokia},
	\item \textit{\textbf{Airbnb, Inc. (ABNB)}} - serwis internetowy umożliwiający wynajem lokali od osób prywatnych\cite{airbnb},
	\item \textit{\textbf{Spotify Technology S.A. (SPOT)}} - szwedzki serwis strumieniowy oferujący dostęp do muzyki oraz podcastów na licencji freemium\cite{spotify},
	\item \textit{\textbf{American International Group, Inc. (AIG)}} - jedno z największych na świecie przedsiębiorstw ubezpieczeniowych\cite{aig}.
\end{itemize}

Na początek sprawdźmy, jak portfel zawierający wymienione aktywa ryzykowne zachowuje się dla rocznej stopy zwrotu aktywa wolnego od ryzyka $r_f = 2\%$, średniej ceny zamknięcia liczonej z jednego dnia oraz zmieniającego się współczynnika $k$ awersji do ryzyka inwestora. Wykres oczekiwanego zwrotu i odchylenia standardowego aktywów ryzykownych dla takiego przypadku przedstawiono na rysunku 6.1.
Zauważmy, że wartością mocno odstającą swoim oczekiwanym zwrotem jak i odchyleniem standardowym jest aktywo ryzykowne spółki \textit{Tesla}, inną wartością jednak mniej wyróżniającą się jest aktywo \textit{Tilray}.
Następnie za pomocą programu otrzymano wagi portfela styczności oraz wagi aktywów w portfelu optymalnym inwestora. Wyniki zapisano w tabelach 6.1 oraz 6.2.\newline
Legenda do tabel:\\
$w_T$ - waga portfela styczności (aktywów ryzykownych) w portfelu optymalnym,\\
$w_{r_f}$ - waga aktywa wolnego od ryzyka w portfelu optymalnym.\\
\begin{table}[ht]
	\centering
	\caption{Wartości wag aktywów ryzykownych w portfelu styczności przy zmieniającym się współczynniku $k$ oraz $r_f = 2\%$ i średnich jednodniowych dla danych z 500 dni.}
	\begin{tabular}{|l|l|l|l|l|l|l|l|l|l|l|}
		\hline
		\textbf{symbol} & \textbf{TSLA} & \textbf{BTI} & \textbf{FB} & \textbf{NFLX} & \textbf{NOK} & \textbf{TLRY} & \textbf{ADDYY} & \textbf{ABNB} & \textbf{SPOT} & \textbf{AIG} \\ \hline
		\textbf{waga}   & 0.12       & 0.22          & 0.58           & -0.11        & -0.01         & 0.01         & 0.14        & -0.06          & 0.30          & -0.19        \\ \hline
	\end{tabular}
\end{table}

\begin{table}[ht]
	\centering
	\caption{Wartości wag portfela optymalnego przy zmieniającym się współczynniku $k$ oraz $r_f = 2\%$ i średnich jednodniowych dla danych z 500 dni.}
	\begin{tabular}{|l|l|l|}
		\hline
		\textbf{$k$} & \textbf{$w_T$} & \textbf{$w_{r_f}$} \\ \hline
		\textbf{0,05}                                             & 0,30           & 0,70            \\ \hline
		\textbf{0,10}                                              & 0,15          & 0,85           \\ \hline
		\textbf{0,15}                                             & 0,10           & 0,90           \\ \hline
		\textbf{0,20}                                              & 0,08          & 0,92           \\ \hline
		\textbf{0,25}                                             & 0,06          & 0,94           \\ \hline
		\textbf{0,30}                                              & 0,05          & 0,95           \\ \hline
		\textbf{0,35}                                             & 0,04          & 0,96           \\ \hline
		\textbf{0,40}                                              & 0,04          & 0,96           \\ \hline
		\textbf{0,45}                                             & 0,03          & 0,97           \\ \hline
		\textbf{0,50}                                              & 0,03          & 0,97           \\ \hline
	\end{tabular}
\end{table}

\begin{figure}[ht]
	\centering
	\includegraphics[scale=0.35]{scatterplot}
	\caption{Wykres oczekwianych zwrotów i odchylenia standardowego aktywów ryzykownych rozpatrywanego portfela.}
\end{figure}
\newpage
Jak zauważamy, już dla wartości współczynnika $k$ na poziomie wartości $0,5$ program sugeruje minimalny wkład pieniężny w aktywa ryzykowne - dzieje się tak dlatego, że wprowadzone przez nas aktywo wolne od ryzyka oferuje stosunkowo wysoki zwrot przy zerowym ryzyku w porównaniu do aktywów ryzykownych (portfel aktywów ryzykownych oferuje oczekiwany zwrot $\mu \approx 26\%$ przy odchyleniu standardowym wynoszącym $\sigma \approx 289\%$) - nawet dla skłonnego do racjonalnego ryzyka inwestora (portfel dla $k = 0,05$) inwestycja w aktywa ryzykowne to tylko 30 procent jego budżetu. Nie będziemy sprawdzać przypadków dla różnej stopy do ryzyka - zmiany w ujęciu dziennym są na tyle niskie, że nie wprowadzają zmiany w obliczanych wagach portfela do drugiego miejsca po przecinku. Sprawdźmy, jak portfel będzie się zachowywać dla liczonej średniej ceny zamknięcia dla średniej tygodniowej oraz miesięcznej. Wyniki dla tak skonstruowanych portfeli przedstawiono w tabelach 6.3, 6.4, 6.5 oraz 6.6.
\begin{table}[ht]
	\centering
	\caption{Wartości wag aktywów ryzykownych w portfelu styczności przy zmieniającym się współczynniku $k$ oraz $r_f = 2\%$ i średnich tygodniowych dla danych z 500 dni.}
	\begin{tabular}{|l|l|l|l|l|l|l|l|l|l|l|}
		\hline
		\textbf{symbol} & \textbf{TSLA} & \textbf{BTI} & \textbf{FB} & \textbf{NFLX} & \textbf{NOK} & \textbf{TLRY} & \textbf{ADDYY} & \textbf{ABNB} & \textbf{SPOT} & \textbf{AIG} \\ \hline
		\textbf{waga}   & -1.51       & 0.50          & -0.75           & 0.51        & 0.06         & -0.36         & 0.47        & 1.51          & 0.03          & 1.48        \\ \hline
	\end{tabular}
\end{table}
\begin{table}[ht]
	\centering
	\caption{Wartości wag portfela optymalnego przy zmieniającym się współczynniku $k$ oraz $r_f = 2\%$ i średnich tygodniowych dla danych z 500 dni.}
	\begin{tabular}{|l|l|l|}
		\hline
		\textbf{$k$} & \textbf{$w_T$} & \textbf{$w_{r_f}$} \\ \hline
		\textbf{0,05}                                             & -0,42         & 1,42           \\ \hline
		\textbf{0,10}                                             & -0,21         & 1,21           \\ \hline
		\textbf{0,15}                                             & -0,14         & 1,14           \\ \hline
		\textbf{0,20}                                             & -0,10         & 1,10           \\ \hline
		\textbf{0,25}                                             & -0,08         & 1,08           \\ \hline
		\textbf{0,30}                                             & -0,06         & 1,06           \\ \hline
		\textbf{0,35}                                             & -0,06         & 1,06           \\ \hline
		\textbf{0,40}                                             & -0,05         & 1,05           \\ \hline
		\textbf{0,45}                                             & -0,05         & 1,05           \\ \hline
		\textbf{0,50}                                             & -0,04         & 1,04           \\ \hline
	\end{tabular}
\end{table}
\begin{table}[ht]
	\centering
	\caption{Wartości wag aktywów ryzykownych w portfelu styczności przy zmieniającym się współczynniku $k$ oraz $r_f = 2\%$ i średnich miesięcznych dla danych z 500 dni.}
	\begin{tabular}{|l|l|l|l|l|l|l|l|l|l|l|}
		\hline
		\textbf{symbol} & \textbf{TSLA} & \textbf{BTI} & \textbf{FB} & \textbf{NFLX} & \textbf{NOK} & \textbf{TLRY} & \textbf{ADDYY} & \textbf{ABNB} & \textbf{SPOT} & \textbf{AIG} \\ \hline
		\textbf{waga}   & 0.35       & 0.46          & 0.60           & -1.17        & -0.20         & 0.32         & 0.57        & 0.81          & -0.07          & -0.65        \\ \hline
	\end{tabular}
\end{table}
\begin{table}[ht!]
	\centering
	\caption{Wartości wag portfela optymalnego przy zmieniającym się współczynniku $k$ oraz $r_f = 2\%$ i średnich miesięcznych dla danych z 500 dni.}
	\begin{tabular}{|l|l|l|}
		\hline
		\textbf{$k$} & \textbf{$w_T$} & \textbf{$w_{r_f}$} \\ \hline
		\textbf{0,05}                                             & 0,22          & 0,78           \\ \hline
		\textbf{0,10}                                             & 0,11          & 0,89           \\ \hline
		\textbf{0,15}                                             & 0,07          & 0,93           \\ \hline
		\textbf{0,20}                                             & 0,06          & 0,94           \\ \hline
		\textbf{0,25}                                             & 0,04          & 0,96           \\ \hline
		\textbf{0,30}                                             & 0,04          & 0,96           \\ \hline
		\textbf{0,35}                                             & 0,03          & 0,97           \\ \hline
		\textbf{0,40}                                             & 0,03          & 0,97           \\ \hline
		\textbf{0,45}                                             & 0,02          & 0,98           \\ \hline
		\textbf{0,50}                                             & 0,02          & 0,98           \\ \hline
	\end{tabular}
\end{table}

\newpage
Wyniki (choć w każdym przypadku przeważające na korzyść aktywa wolnego od ryzyka) są dość rozbieżne, mimo tego że rozpatrujemy wciąż portfel tych samych aktywów ryzykownych. Według Markowitza, zwroty aktywów powinny mieć rozkład normalny - sprawdźmy, jak wyglądają rozkłady zwrotów wybranych przez nas spółek - wynik przedstawiono na rysunku 6.2.

\begin{figure}[ht]
	\centering
	\includegraphics[scale=0.3]{zwrotyplot}
	\caption{Rozkład zwrotów aktywów ryzykownych rozpatrywanego portfela.}
\end{figure}

Na otrzymanym wykresie możemy zauważyć charakterystyczny "dzwon" w okolicy wartości $0$ na osi $x$. Świadczy to o tym, że problemem nie jest nieprzystający do założeń rozkład średnich. Powodem może być drugie z założeń Markowitza - mówiące o racjonalności inwestorów. W rzeczywistości rynek jest narażony na duże zmiany wynikające z chwilowego skupienia inwestorów na danej spółce lub grupie spółek\cite{simplifiedmarkowitz}. Bardzo jaskrawym przykładem braku racjonalności inwestorów może być przypadek z początku pandemii koronawirusa, kiedy znacząco wzrósł popyt na oprogramowanie komputerow umożliwiające zdalne przeprowadzanie spotkań - w szczególności \textit{Microsoft Teams} oraz \textit{Zoom}. Inwestorzy mylnie zaczęli inwestować w aktywa \textit{Zoom Technologies} - firmy, która nie miała nic wspólnego z komunikatorem \textit{Zoom} - doprowadziło to do wzrostu ceny akcji o 1800\% względem stanu z początku roku, gdzie akcje firmy zajmującej się komunikatorem \textit{Zoom} wzrosły jedynie o 132\%. W konsekwencji doprowadziło to do zablokowania inwestycji w aktywa \textit{Zoom Technologies}.\cite{zoominvest}

\begin{figure}[ht]
	\centering
	\includegraphics[scale=0.3]{zoom}
	\caption{Graficzny opis sytuacji spółek \textit{Zoom} oraz \textit{Zoom Technologies} z pierwszego kwartału 2020 roku. Źródło:\cite{zoominvest}.}
\end{figure}
\newpage
Aby lepiej pokazać narażenie rynku na zmiany, zmieńmy liczbę dni z których pobierane są dane do 1000, następnie przeliczmy ponownie wagi portfela optymalnego dla zmieniającego się współczynnika $k$, $r_f = 2\%$ i średnich dziennych, tygodniowych oraz miesięcznych. Wyniki przedstawiono w tabelach od 6.7 do 6.12.
\begin{table}[ht]
	\centering
	\caption{Wartości wag aktywów ryzykownych w portfelu styczności przy zmieniającym się współczynniku $k$ oraz $r_f = 2\%$ i średnich jednodniowych dla danych z 1000 dni.}
	\begin{tabular}{|l|l|l|l|l|l|l|l|l|l|l|}
		\hline
		\textbf{symbol} & \textbf{TSLA} & \textbf{BTI} & \textbf{FB} & \textbf{NFLX} & \textbf{NOK} & \textbf{TLRY} & \textbf{ADDYY} & \textbf{ABNB} & \textbf{SPOT} & \textbf{AIG} \\ \hline
		\textbf{waga}   & 0.15       & -0.90          & 0.46           & 0.63        & -0.02         & 0.18         & 1.00        & -0.13          & 0.06          & -0.43        \\ \hline
	\end{tabular}
\end{table}
\begin{table}[ht]
	\centering
	\caption{Wartości wag portfela optymalnego przy zmieniającym się współczynniku $k$ oraz $r_f = 2\%$ i średnich jednodniowych dla danych z 1000 dni.}
	\begin{tabular}{|l|l|l|}
		\hline
		\textbf{$k$} & \textbf{$w_T$} & \textbf{$w_{r_f}$} \\ \hline
		\textbf{0,05}                                             & 0,23          & 0,77           \\ \hline
		\textbf{0,10}                                             & 0,11          & 0,89           \\ \hline
		\textbf{0,15}                                             & 0,08          & 0,92           \\ \hline
		\textbf{0,20}                                             & 0,06          & 0,94           \\ \hline
		\textbf{0,25}                                             & 0,05          & 0,95           \\ \hline
		\textbf{0,30}                                             & 0,04          & 0,96           \\ \hline
		\textbf{0,35}                                             & 0,03          & 0,97           \\ \hline
		\textbf{0,40}                                             & 0,03          & 0,97           \\ \hline
		\textbf{0,45}                                             & 0,03          & 0,97           \\ \hline
		\textbf{0,50}                                             & 0,02          & 0,98           \\ \hline
	\end{tabular}
\end{table}
\begin{table}[ht]
	\centering
	\caption{Wartości wag aktywów ryzykownych w portfelu styczności przy zmieniającym się współczynniku $k$ oraz $r_f = 2\%$ i średnich tygodniowych dla danych z 1000 dni.}
	\begin{tabular}{|l|l|l|l|l|l|l|l|l|l|l|}
		\hline
		\textbf{symbol} & \textbf{TSLA} & \textbf{BTI} & \textbf{FB} & \textbf{NFLX} & \textbf{NOK} & \textbf{TLRY} & \textbf{ADDYY} & \textbf{ABNB} & \textbf{SPOT} & \textbf{AIG} \\ \hline
		\textbf{waga}   & -0.30       & 0.35          & 0.12           & -0.19        & 0.19         & -0.26         & -0.29        & 0.80          & -0.14          & 0.71        \\ \hline
	\end{tabular}
\end{table}
\begin{table}[ht!]
	\centering
	\caption{Wartości wag portfela optymalnego przy zmieniającym się współczynniku $k$ oraz $r_f = 2\%$ i średnich tygodniowych dla danych z 1000 dni.}
	\begin{tabular}{|l|l|l|}
		\hline
		\textbf{$k$} & \textbf{$w_T$} & \textbf{$w_{r_f}$} \\ \hline
		\textbf{0,05}                                             & -2,41         & 3,41            \\ \hline
		\textbf{0,10}                                             & -1,21         & 2,21            \\ \hline
		\textbf{0,15}                                             & -0,80         & 1,80            \\ \hline
		\textbf{0,20}                                             & -0,60         & 1,60            \\ \hline
		\textbf{0,25}                                             & -0,48         & 1,48            \\ \hline
		\textbf{0,30}                                             & -0,40         & 1,40            \\ \hline
		\textbf{0,35}                                             & -0,34         & 1,34            \\ \hline
		\textbf{0,40}                                             & -0,30         & 1,30            \\ \hline
		\textbf{0,45}                                             & -0,27         & 1,27            \\ \hline
		\textbf{0,50}                                             & -0,24         & 1,24            \\ \hline
	\end{tabular}
\end{table}
\begin{table}[ht!]
	\centering
	\caption{Wartości wag aktywów ryzykownych w portfelu styczności przy zmieniającym się współczynniku $k$ oraz $r_f = 2\%$ i średnich miesięcznych dla danych z 1000 dni.}
	\begin{tabular}{|l|l|l|l|l|l|l|l|l|l|l|}
		\hline
		\textbf{symbol} & \textbf{TSLA} & \textbf{BTI} & \textbf{FB} & \textbf{NFLX} & \textbf{NOK} & \textbf{TLRY} & \textbf{ADDYY} & \textbf{ABNB} & \textbf{SPOT} & \textbf{AIG} \\ \hline
		\textbf{waga}   & 0.05       & 0.14          & 0.31           & 0.00        & 0.35         & -0.04         & 0.40        & 0.32          & -0.17          & -0.38        \\ \hline
	\end{tabular}
\end{table}
\begin{table}[ht]
	\centering
	\caption{Wartości wag portfela optymalnego przy zmieniającym się współczynniku $k$ oraz $r_f = 2\%$ i średnich miesięcznych dla danych z 1000 dni.}
	\begin{tabular}{|l|l|l|}
		\hline
		\textbf{$k$} & \textbf{$w_T$} & \textbf{$w_{r_f}$} \\ \hline
		\textbf{0,05}                                             & -1,29         & 1,29           \\ \hline
		\textbf{0,10}                                             & -0,65         & 1,65           \\ \hline
		\textbf{0,15}                                             & -0,43         & 1,43           \\ \hline
		\textbf{0,20}                                             & -0,32         & 1,32           \\ \hline
		\textbf{0,25}                                             & -0,26         & 1,26           \\ \hline
		\textbf{0,30}                                             & -0,22         & 1,22           \\ \hline
		\textbf{0,35}                                             & -0,18         & 1,18           \\ \hline
		\textbf{0,40}                                             & -0,16         & 1,16           \\ \hline
		\textbf{0,45}                                             & -0,14         & 1,14           \\ \hline
		\textbf{0,50}                                             & -0,13         & 1,13           \\ \hline
	\end{tabular}
\end{table}
\newpage
\textcolor{white}{.}
\newpage
Przedstawione portfele (poza portfelami w tabeli 6.8) uwydatniają problemy założenia o nieskończonym dostępie do kapitału.\cite{simplifiedmarkowitz} Wagi, a przez to pieniądze które powinny zostać zainwestowane w dane aktywa, są na tyle duże że przewyższają dostępny bużet.
Nauczeni wnioskami z poprzednich operacji, spróbujmy uwydatnić nasz portfel przez lepszy dobór aktywów ryzykownych - tak, aby stworzyć przykład funduszu, który dla różnego współczynnika $k$ oraz zmiany liczonej średniej ceny zamknięcia będzie oferować wciąż racjonalne wagi. Pierwszym krokiem będzie pozbycie sie aktywów będących wartościami odstającymi w portfelu - jest to przytoczona wcześniej \textit{Tesla} oraz \textit{Tilray} (patrz rysunek 6.1). Oprócz tego zdecydowano się zrezygnować z aktywów \textit{British American Tobacco}. Zamiast tego, skorzystano ze aktywów spółek należących do indeksu akcji \textit{Dow Jones} - jest to indeks składający się z trzydziestu wielkich amerykańskich spółek. Wybrano następujące aktywa do portfela:

\begin{itemize}
	\item \textit{\textbf{Apple (AAPL)}} - amerykańskie przedsiębiorstwo informatyczne z siedzibą w Cupertino w Kalifornii,\cite{apple}
	\item \textit{\textbf{Microsoft Corporation (MSFT)}} - amerykańskie przedsiębiorstwo informatyczne,\cite{microsoft}
	\item \textit{\textbf{Coca-Cola Company (KO)}} -  jeden z największych na świecie producentów, dystrybutorów i sprzedawców napojów bezalkoholowych.\cite{coke}
\end{itemize}

Tak skonstruowany portfel przetestowano podobnie jak w poprzednim przykładzie na różnych stopniach współczynnika $k$, średnich jednodniowych oraz rocznej stopie zwrotu aktywa wolnego od ryzyka $r_f = 0,1\%$.
Wyniki inwestycji w aktywa ryzykowne (wagi portfela styczności) przedstawiono w tabeli 6.13, natomiast wagi portfela optymalnego w tabeli 6.14.

\begin{table}[ht]
	\centering
	\caption{Wartości wag portfela styczności przy zmieniającym się współczynniku $k$ oraz $r_f = 0,1\%$ i średnich jednodniowych dla danych z 1000 dni.}
	\begin{tabular}{|l|l|l|l|l|l|l|l|l|l|l|}
		\hline
		\textbf{symbol} & \textbf{FB} & \textbf{NFLX} & \textbf{ADDYY} & \textbf{NOK} & \textbf{ABNB} & \textbf{SPOT} & \textbf{AIG} & \textbf{AAPL} & \textbf{MSFT} & \textbf{KO} \\ \hline
		\textbf{waga}   & -0.16       & 0.13          & 0.40           & -0.15        & -0.01         & -0.08         & -0.33        & 0.05          & 1.13          & 0.02        \\ \hline
	\end{tabular}
\end{table}

\begin{table}[ht]
	\centering
	\caption{Wartości wag portfela optymalnego przy zmieniającym się współczynniku $k$ oraz $r_f = 2\%$ i średnich jednodniowych dla danych z 1000 dni.}
	\begin{tabular}{|l|l|l|}
		\hline
		\textbf{$k$} & \textbf{$w_T$} & \textbf{$w_{r_f}$} \\ \hline
		\textbf{0,05} & 0,42          & 0,58           \\ \hline
		\textbf{0,10} & 0,21          & 0,79           \\ \hline
		\textbf{0,15} & 0,14          & 0,86           \\ \hline
		\textbf{0,20} & 0,10          & 0,90           \\ \hline
		\textbf{0,25} & 0,08          & 0,92           \\ \hline
		\textbf{0,30} & 0,07          & 0,93           \\ \hline
		\textbf{0,35} & 0,06          & 0,94           \\ \hline
		\textbf{0,40} & 0,05          & 0,95           \\ \hline
		\textbf{0,45} & 0,05          & 0,95           \\ \hline
		\textbf{0,50} & 0,04          & 0,96           \\ \hline
	\end{tabular}
\end{table}

Jak zauważamy, pomimo teoretycznie lepszego doboru spółek do portfela, wciąż borykamy się z realiami - połowa spółek w portfelu aktywów ryzykownych ma wagi ujemne - podjęte zostały próby doboru innych spółek, jednak bez wyraźnie lepszego rezulatu. Wysoka waga aktywów spółki \textit{Microsoft} nie dziwi - na rysunku 6.4 umieszczono ceny akcji z ostatnich 5 lat, gdzie (poza przypadającym na pierwszy kwartał 2020 początkiem pandemii koronawirusa) nie występują drastyczne spadki wartości aktywów, a zauważalny jest stabilny wzrost wartości.

\begin{figure}[ht]
	\centering
	\includegraphics[width=\textwidth]{msftakcje}
	\caption{Ceny akcji firmy Microsoft(MSFT) w okresie 17.05.2016-17.05.2021,\newline
		źródło:https://www.bankier.pl/inwestowanie/profile/quote.html?symbol=MSFT.}
\end{figure}


\chapter{Wnioski oraz plany rozwoju aplikacji}

Portfel Markowitza to narzędzie dobrze sprawdzające się w warunkach teoretycznych, jednak w zderzeniu z rzeczywistością zauważalne są jego ograniczenia, spowodowane założeniami teorii. Wspomniano już o racjonalności inwestorów, nie mającej zawsze przełożenia na realia rynkowe, oraz nielimitowany dostęp do kapitału. Innym ważnym aspektem, który wpływa na działanie portfela jest to, nie istnieje w aktywo całkowicie wolne od ryzyka - przykładowo obligacje skarbu państwa faktycznie są wolne od ryzyka, jednak inflacja oraz zmiana stóp procentowych może wpływać na ich wartość.\cite{hip} Oprócz tego, teoria Markowitza zakłada, że rynek jest doskonale wydajny \cite{markowitz} - jednak jak widzimy rynek jest podatny na różne czynniki z zewnątrz (osobiste, środowiskowe i tym podobne), a nie jedynie na jego cenę historyczną. Powoduje to, że rynek daleki jest od bycia w pełni wydajnym.
Zauważalna jest potrzeba odpowiedniego doboru aktywów do portfela - korzystając z indeksu \textit{Dow Jones} do doboru spółek, udało się osiągnąć bardziej zadowalające rezultaty jeżeli chodzi o wagi portfela niż przez dodanie spółek według uznania.
Pomimo wszystkich niedociągnięć narzędzia, trzeba pamiętać że znalazło ono uznanie w świecie finansów i pomimo tego, że zostało sformułowane prawie 60 lat temu, wciąż znajdują się chętni na rozwinięcie idei Harry'ego Markowitza - przykładami są między innymi \textit{Post-Modern Portfolio Theory}, sformułowany przez Briana M. Roma oraz Kathleen Ferguson, opierający się na ryzyku spadku wartości akcji zamiast na średniej wartości zwrotów\cite{pmpt} oraz wspomniany już w tej pracy model wyceny aktywów kapitałowych (\textit{CAPM})\cite{holton}. Jak przyznał sam Harry Markowitz w uwagach do swojej pracy \textit{Portfolio Selection}, "To historia której przeczytałem jedynie pierwszą stronę pierwszego rozdziału"\cite{markowitz}.
\\
W sprawie samego programu - jego działanie było jak najbardziej prawidłowe i wystarczające na potrzeby stworzenia tej pracy magisterskiej. Aplikacja spełniła założenia projektowe, jednak istnieją plany jej rozwoju. W następnych wersjach planowana jest:
\begin{itemize}
	\item możliwość dodania oraz usunięcia spółki w trybie \textit{file-mode},
	\item możliwość ukazania wyników w postaci wykresu słupkowego,
	\item opcja wizualizacji rozkładu zwrotów aktywów ryzykownych w portfelu,
	\item optymalizacja programu pod kątem szybkości działania (przechowywanie danych zamiast pobierania ich przy każdej operacji przeliczania portfela),
	\item opcja zapisu wyników portfela optymalnego w przyjaznej użytkownikowi formie do arkusza programu \textit{Microsoft Excel}.
\end{itemize}

\appendixpage
\appendix
%\addappheadtotoc

\chapter{Dodatek - pełny kod programu}\label{Dod1}

\section{main.py}
\inputminted[breaklines]{python}{the-wallet/main.py}

\section{functions.py}
\inputminted[breaklines]{python}{the-wallet/functions.py}

\section{additional\_functions.py}
\inputminted[breaklines]{python}{the-wallet/additional_functions.py}

\bibliographystyle{dyplom1}

\begin{thebibliography}{30}
	\bibitem{hip}
	B. McClure,
	\emph{Modern portfolio theory: Why it’s still hip.},
	Investopedia, 2010.
	
	\bibitem{pmpt}
	Brian M. Rom, Kathleen W. Ferguson,
	\emph{Post-Modern Portfolio Theory Comes of Age},
	The Journal of Investing, 1993.

	\bibitem{determineparameters}
	Ertugrul Bayraktar, Ayse Humeyra Bilge
	\emph{Determination the Parameters of Markowitz Portfolio Optimization Model},
	ArXiv, 2013.

	\bibitem{holton}
	Glyn Holton,
	\emph{Portfolio Theory},
	https://www.glynholton.com/notes/portfolio\_theory/, 2013.

	\bibitem{zoominvest}
	\emph{GroupChat Newsletter \#149},\\
	https://us4.campaign-archive.com/?u=7edf930a619a472beb2d9346\&id=ad40099241.

	\bibitem{markowitz}
	H. Markowitz,
	\emph{Portfolio Selection},
	The Journal of Finance 7(1), 1952.
	
	\bibitem{book}	
	Michael A. Bean,
	\emph{Probability: The Science of Uncertainty: with Applications to Investments, Insurance, and Engineering},
	American Mathematical Society, 2001.

	\bibitem{simplifiedmarkowitz}
	Myles Mangram,
	\emph{A Simplified Perspective of the Markowitz Portfolio Theory},
	Global Journal of Business Research, 2013.
	
	\bibitem{pythonlist}
	Python Software Foundation,
	\emph{Documentation -> The Python Tutorial -> Data Structures},
	https://docs.python.org/3/tutorial/datastructures.html.
		
	\bibitem{birdtippett}
	Ron Bird, Mark Tippett,
	\emph{Note---Naive Diversification and Portfolio Risk---A Note},
	Management Science, 1986.

	\bibitem{brandsgallagher}
	Simone Brands, David R. Gallagher,
	\emph{Portfolio Concentration and Investment Performance},
	NYU Working Paper No., 2004.

	\bibitem{lagrange}
	Uniwersytet Warszawski,
	\emph{Ekstrema Warunkowe i Mnożniki Lagrange'a},
	http://dydmat.mimuw.edu.pl/analiza-matematyczna-ii/ekstrema-warunkowe-i-mnozniki-la\newline
	grangea.

	\bibitem{howmanystocks}
	V. Alexeev, F.Tapon,
	\emph{Equity Portfolio Diversification: How Many Stocks are Enough? Evidence from Five Developed Markets},
	Advanced Risk \& Portfolio Management® Research Paper Series, 2012.

	\bibitem{wikipage}
	Wikipedia,
	\emph{Teoria Portfelowa},
	https://pl.wikipedia.org/wiki/Teoria\_portfelowa.
	
	\bibitem{random_variable}
	Wikipedia,
	\emph{Zmienna losowa},
	https://pl.wikipedia.org/wiki/Zmienna\_losowa.
	
	\bibitem{expected_value}
	Wikipedia,
	\emph{Wartość oczekiwana},
	https://pl.wikipedia.org/wiki/Wartość\_oczekiwana.

	\bibitem{tesla}
	Wikipedia,
	\emph{Tesla (amerykańskie przedsiębiorstwo)},\\
	https://pl.wikipedia.org/wiki/Tesla\_(ameryka\%C5\%84skie\_przedsi\%C4\%99biorstwo).
	
	\bibitem{facebook}
	Wikipedia,
	\emph{Facebook (przedsiębiorstwo)},\\
	https://pl.wikipedia.org/wiki/Facebook\_(przedsi\%C4\%99biorstwo).
	
	\bibitem{netflix}
	Wikipedia,
	\emph{Netflix},
	https://pl.wikipedia.org/wiki/Netflix.
	
	\bibitem{tilray}
	Wikipedia,
	\emph{Tilray},
	https://en.wikipedia.org/wiki/Tilray.
	
	\bibitem{adidas}
	Wikipedia,
	\emph{Adidas},
	https://pl.wikipedia.org/wiki/Adidas.
	
	\bibitem{nokia}
	Wikipedia,
	\emph{Nokia},
	https://pl.wikipedia.org/wiki/Nokia.
	
	\bibitem{airbnb}
	Wikipedia,
	\emph{Airbnb},	
	https://pl.wikipedia.org/wiki/Airbnb.
	
	\bibitem{spotify}
	Wikipedia,
	\emph{Spotify},
	https://pl.wikipedia.org/wiki/Spotify.
	
	\bibitem{aig}
	Wikipedia,
	\emph{American International Group},\\
	https://pl.wikipedia.org/wiki/American\_International\_Group.
	
	\bibitem{apple}
	Wikipedia,
	\emph{Apple (przedsiębiorstwo)},\\
	https://pl.wikipedia.org/wiki/Apple\_(przedsiębiorstwo).
	
	\bibitem{microsoft}
	Wikipedia,
	\emph{Microsoft},\\
	https://pl.wikipedia.org/wiki/Microsoft.
	
	\bibitem{coke}
	Wikipedia,
	\emph{The Coca-Cola Company},\\
	https://pl.wikipedia.org/wiki/The\_Coca-Cola\_Company.
\end{thebibliography}
\begin{comment}
Można więc stworzyć układ równań:

$$
\left\{ \begin{array}{l}
2{\sigma_1}^2w_1 + 2\sigma_{12}w_2 - \tau(\mu_1 - r_f) = 0\\
2{\sigma_2}^2w_2 + 2\sigma_{12}w_1 - \tau(\mu_2 - r_f) = 0\\
(\mu_1 - r_f)w_1 + (\mu_2 - r_f)w_2 - \mu_p + r_f = 0
\end{array} \right.
$$

Przekształcając pierwsze dwa równania otrzymujemy:

$$
\left\{ \begin{array}{l}
{\sigma_1}^2w_1 + \sigma_{12}w_2 = \frac{\tau}{2}(\mu_1 - r_f)\\
{\sigma_2}^2w_2 + \sigma_{12}w_1 = \frac{\tau}{2}(\mu_2 - r_f)\\
(\mu_1 - r_f)w_1 + (\mu_2 - r_f)w_2 - \mu_p + r_f = 0
\end{array} \right.
$$



\begin{equation}
\left( \begin{array}{cc}
{\sigma_1}^2 & \sigma_{12}\\
\sigma_{12} & {\sigma_2}^2
\end{array} \right)
\left( \begin{array}{c}
w_1\\
w_2
\end{array} \right) =
(\frac{\tau}{2})
\left( \begin{array}{c}
\mu_1 - r_f\\
\mu_2 - r_f
\end{array} \right)
\end{equation}
\end{comment}
\end{document}
