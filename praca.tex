% !TeX spellcheck = <none>
\documentclass[magister]{dyplom}
\usepackage[utf8]{inputenc}
\usepackage{hyperref}
\usepackage{verbatim}
\usepackage{minted}
\usepackage{lmodern}
\usepackage{subfig}
%%
\usepackage[toc]{appendix}
\renewcommand{\appendixtocname}{Dodatki}
\renewcommand{\appendixpagename}{Dodatki}
%%
\usepackage{listings}

%\noappendicestocpagenum
\usepackage{lipsum}
\usepackage{url}
%\usepackage{showframe}
\def\code#1{\texttt{#1}}

\author{Radosław Majchrzak}\album{178640}
\title{Narzędzie komputerowe wyznaczające optymalny portfel dla inwestora – portfel Markowitza}
\titlen{}
\promotor{dr hab. inż. prof. Uczelni Zbigniew Michna}
\kierunek{Informatyka w Biznesie}
\specjalnosc{Big Data}
\date{}

\begin{document}

\maketitle

\tableofcontents


\chapter{Cel i zakres pracy}

W roku 1952 amerykański ekonomista Harry Markowitz w artykule \textit{Portfolio Selection} opublikował jedną z najważniejszych teorii w świecie finansów, zwaną teorią portfelową, lub od nazwiska twórcy teorii - Portfelem Markowitza. Był to pierwszy model portfela inwestora, w którym został jawnie wprowadzony parametr ryzyka w inwestycji kapitału\cite{wikipage}. Pokrótce mówiąc, rozwiązanie zaproponowane przez autora artykułu ma maksymalizować zysk inwestora przy danym stopniu ryzyka lub minimalizować ryzyko przy danym stopniu zysku. Teoria ta stała się dla mnie na tyle interesująca, że zdecydowałem się na oparcie na niej mojej pracy magisterskiej.\\\par
Przedmiotem pracy jest stworzenie narzędzia do wyznaczania portfela styczności oraz portfela optymalnego dla danego inwestora (tzn. inwestora z określoną awersją do ryzyka przy wybranej stopie wolnej od ryzyka w jego portfelu oraz wybranych przez niego aktywów ryzykownych) i zbadanie działania tego narzędzia.
Będzie to program napisany w języku komputerowym \textit{Python} w wersji 3.7.8.\\\par
W pracy zawarta jest dokumentacja opisowa narzędzia (zwanego w dalszych częściach pracy \textit{programem} lub \textit{aplikacją}). Wszystkie szczegóły dotyczące rozwiązań technicznych oraz opis matematyczny teorii są umieszczone w odpowiednich rozdziałach i podrozdziałach.\\\par

\chapter{Wprowadzenie matematyczne}
W tym rozdziale zostały krótko wyjaśnione pojęcia matematyczne, które są użyte w tej pracy:\\
\begin{itemize}
\item \textit{zmienna losowa} - jest to funkcja przypisująca zdarzeniom elementarnym liczby\cite{random_variable}\newline 

\item \textit{wartość oczekiwana} - dla zmiennej losowej X typu skokowego jest opisana przez liczbę:
\begin{equation}
	EX = \sum_{n=1}^{\infty}x_np_n,
\end{equation}
gdzie:\\
$p_n$ - rozkład zmiennej losowej ($p_n = P(X = x_n)$)
$x_n$ - pojedyncza obserwacja zmiennej losowej X

Estymatorem wartości oczekiwanej rozkładu zmiennej losowej w populacji jest średnia arytmetyczna $n$ niezależnych obserwacji danej zmiennej, opisana wzorem \cite{expected_value}:
\begin{equation}
	\overline{X} = \frac{1}{n}(X_1 + X_2 + X_3 + \dots + X_n),
\end{equation}
gdzie:\\
$X$ - zmienna będąca obiektem działania\\
$X_n$ - pojedyncza obserwacja zmiennej\newline

\item \textit{wariancja} - miara zmienności danych, oznaczana w matematyce przez znak $\sigma^2$. Określa to, jak bardzo wszystkie możliwe wartości zmiennej są rozproszone względem jej \textit{wartości oczekiwanej}. Dla zmiennej losowej X typu skokowego jest opisana przez liczbę:

\begin{equation}
	D^2(X) = \sum_n(x_n - EX)^2p_n,
\end{equation}

Często używanym estymatorem jest wariancja próbkowa, gdzie wzór przedstawia się następująco:

\begin{equation}
	\sigma^2 = \frac{(X_1 - \overline{X})^2 + (X_2 - \overline{X})^2 + \dots + (X_n - \overline{X})^2}{n}
\end{equation}

\item \textit{odchylenie standardowe} - pierwiastek kwadratowy z wariancji: \newline

\begin{equation}
	D(X) = \sqrt{D^2(X)},
\end{equation}

\begin{equation}
	\sigma = \sqrt{\frac{(X_1 - \overline{X})^2 + (X_2 - \overline{X})^2 + \dots + (X_n - \overline{X})^2}{n}}
\end{equation}
\newpage
\item \textit{kowariacja} - miara relacji między dwiema \textit{zmiennymi losowymi} - może przyjmować dowolne wartości zarówno dodatnie, jak i ujemne, które mają następującą interpretację:
\begin{itemize}
	\item Wartość dodatnia kowariancji - dwie zmienne losowe mają tendencję do ruchu w tym samym kierunku (to znaczy - jak jedna zmienna przyjmuje duże wartości to i druga zmienna losowa przyjmuje duże wartości),
	\item Wartość ujemna kowariancji - dwie zmienne losowe mają tendencję do ruchu w tym przeciwnym kierunku (to znaczy - jak jedna zmienna przyjmuje małe wartości to i druga zmienna losowa przyjmuje małe wartości),
	\item Wartość kowariancji to 0 - dwie zmienne losowe są nieskorelowane.
\end{itemize}

Kowariancja dla próby jest opisana wzorem:
\begin{equation}
	Cov(X,Y) = \frac{\sum(X_i - \overline{X})(Y_j - \overline{Y})}{n - 1}
\end{equation}

\item \textit{korelacja} - miara relacji między dwoma zmiennymi losowymi, wynikiem jest liczba należąca do zbioru $<-1,1>$. Korelacja dla próby jest opisana wzorem:
\begin{equation}
	\rho_{X,Y} = \frac{Cov(X,Y)}{\sigma_X\sigma_Y},
\end{equation}
gdzie:\\
$\sigma_X$ - odchylenie standardowe zmiennej losowej X\\
$\sigma_Y$ - odchylenie standardowe zmiennej losowej Y\\

Interpretacja wartości korelacji przedstawia się następująco:
\begin{itemize}
	\item Wartość dodatnia korelacji- dwie zmienne losowe mają tendencję do ruchu w tym samym kierunku (to znaczy - jeżeli jedna z nich rośnie, to druga też),
	\item Wartość ujemna korelacji - dwie zmienne losowe mają tendencję do ruchu w tym przeciwnym kierunku (to znaczy - jeżeli jedna z nich rośnie, to druga maleje),
	\item Wartość korelacji to 0 - dwie zmienne losowe są od siebie niezależne.
\end{itemize}

Im wartość korelacji jest bliższa wartości skrajnej, tym mocniejsze jest oddziaływanie jednej zmiennej na drugą.\\

\item \textit{rozkład prawdopodobieństwa} - jest to przedstawienie możliwych wartości zmiennej oraz tego, jak często mogą one wystąpić. Zbiór możliwych wartości jest opisany przez \textit{X}, a pojedyncza wartość z tego zbioru przez \textit{x}. Najczęściej występującym w przyrodzie rozkładem prawdopodobieństwa jest \textit{rozkład normalny}.\newline
\newpage
\item \textit{rozkład normalny prawdopodobieństwa} - najczęściej występujący w przyrodzie rozkład prawdopodobieństwa - wykres jest krzywą w kszatłcie dzwonu, widocznego na rysunku 2.1:
\begin{figure}[h]
	\centering
	\includegraphics[scale=0.4]{Standard_Normal_Distribution.png}
	\caption{Wykres rozkładu normalnego,\newline
		źródło:https://commons.wikimedia.org/wiki/File:Standard\_Normal\_Distribution.png}
\end{figure}
\end{itemize}
\chapter{Teoria portfela Markowitza}

\section{Wprowadzenie}

Jak wiadomo, pojedyncze akcje na giełdzie cechują się dość dużą nieprzewidywalnością - ich zachowania można przewidywać, jednak zdarzenia losowe mogą te przewidywania brutalnie zweryfikować. Jako przykład na rysunku 3.1 przedstawiono wykres ceny akcji spółki Nokia w ostatnich 2 latach. Jest on niejednostajny, zauważalne są duże spadki, które przypadały na zdarzenia takie jak choćby ogłoszenie zmiany prezesa firmy oraz początek pandemii koronawirusa. 

\begin{figure}[h]
\includegraphics[width=\textwidth]{nokia}
\caption{Ceny akcji firmy NOKIA(NOK1V) w okresie 21.01.2019-21.01.2021,\newline
	źródło:https://www.bankier.pl/inwestowanie/profile/quote.html?symbol=NOK1V}
\end{figure}

Jak zauważa Markowitz w swoim artykule \textit{Portfolio Selection} - inwestor może uchronić się (w pewnym stopniu) przed niestabilnością rynku przez stworzenie portfela składającego się z różnych aktywów tak, by zminimalizować ryzyko. Minimalizacja ryzyka przez zakup różnych aktywów nazywana jest \textit{dywersyfikacją}. Markowitz w swoim modelu przyjmuje dwa ważne założenia: 

\begin{enumerate}
	\item Zwroty mają rozkład normalny
	\item Inwestorzy są racjonalni i niechętni do podejmowania ryzyka - mogą je podjąć tylko, jeżeli przewidują wyższy zwrot.
\end{enumerate}

Załóżmy, że portfel składa się z ustalonych aktywów. Stopa zwrotu z tego portfela po ustalonym czasie jest zmienną losową o pewnej wartości oczekiwanej i wariancji (zależnej od wag aktywów w tym portfelu).
Punkt drugi założeń Markowitza prowadzi więc do ważnej konkluzji - dla jakiegokolwiek akceptowalnego poziomu ryzyka istnieje tylko jeden portfel, który da najwyższy oczekiwany zwrot (dla oczekiwanych wag osiąga on maksimum) i dla danej średniej (oczekiwanej) stopy zwrotu istnieje tylko jeden portfel z minimalnym ryzykiem (dla odpowiednich wag osiąga ono minimum). Każdy taki portfel nazywany jest \textit{portfelem wydajnym} (lub \textit{efektywnym}).\par

\section{Portfele wydajne dla dwóch aktywów ryzykownych} 

Załóżmy, że na rynku dostępne są dwa walory ryzykowne. Nasz portfel składa się z dwóch takich aktywów, $S_1$ oraz $S_2$. Zmienne losowe $R_1$ oraz $R_2$ opisują stopę zwrotu z inwestycji w te aktywa po ustalonym okresie. Korelacja pomiędzy stopami zwrotu inwestycji wynosi $\rho$. Zadanie to znalezienie portfeli wydajnych - to znaczy takich proporcji tych dwóch walorów, dla których powinno się otrzymać największy średni zysk przy ustalonym ryzyku.\\
Oczekiwany zwrot z portfela po danym okresie będzie wynosić:

\begin{equation}
	R = w_1R_1 + w_2R_2,
\end{equation}
gdzie $w_n$ - waga aktywa n w portfelu, a $R$ - oczekiwany zwrot z całego portfela.\\
$R$, podobnie jak $R_1$ oraz $R_2$ jest zmienną losową. Mając na uwadze założenie Markowitza, że zwroty mają rozkład normalny, wspólny rozkład prawdopodobieństwa wszystkich tych zmiennych (tj. $R$, $R_1$, $R_2$) jest dwuwymiarowym rozkładem normalnym - stąd są one całkowicie scharakteryzowane przez swoje wartości średniej ($\mu_n$) i odchylenia standardowego ($\sigma_n$). W takim przypadku wzór na oczekiwaną stopę zwrotu aktywów $S_1$ oraz $S_2$ opisuje równanie 3.2, a wariancję równanie 3.3:

\begin{equation}
	\mu = w_1\mu_1 + w_2\mu_2,
\end{equation}
gdzie:\\
$w_n$ - waga aktywa n w portfelu\\
$\mu$ - średni zwrot z całego portfela.\\
\begin{equation}
	\sigma^2 = w_1^{2}\sigma_1^{2} + w_2^{2}\sigma_2^{2} + 2w_1\sigma_1w_2\sigma_2\rho, 
\end{equation}
gdzie:\\
$w_n$ - waga aktywa n w portfelu\\
$\sigma^2$ - wariancja całego portfela.\\
Wagi walorów w portfelu muszą sumować się do 1. Wiedząc, że portfel składa się z dwóch walorów:
\begin{equation}
	w_1 + w_2 = 1,
\end{equation}
przy czym wartości wag, niezależnie od ilości aktywów nie są ograniczone do zakresu $<0,1>$ - ujemna waga waloru oznacza, że aktywo takie staje się w portfelu \textit{pozycją krótką}. Oznacza to, że pożyczamy to aktywo, sprzedajemy i inwestujemy wszystko co mamy w drugie aktywo.\\
Wzory 3.2 oraz 3.3 można uprościć, jako $w_2$ podstawiając $1 - w_1$, dzięki czemu pozbywamy się jednej niewiadomej. Rezultatem jest:

\begin{equation}
	\mu = w_1\mu_1 + (1 - w_1)\mu_2 ,
\end{equation}

\begin{equation}
	\sigma^2 = w_1^{2}\sigma_1^{2} + (1 - w_1)^{2}\sigma_2^{2} + 2w_1\sigma_1(1 - w_1)\sigma_2\rho.
\end{equation}
\\
Wyznaczając wagę $w_1$ z równania 3.5, otrzymujemy:

\begin{equation}
	w_1 = \frac{\mu - \mu_2}{\mu_1 - \mu_2},
\end{equation}
a podstawiając otrzymaną w równaniu 3.7 wagę $w_1$ do równania 3.6 otrzymujemy:

\begin{equation}
	\sigma^2 = \frac{(\mu - \mu_2)^2}{(\mu_1 - \mu_2)^2}\sigma_1^{2} + 2\frac{(\mu - \mu_2)(\mu_1 - \mu)}{(\mu_1 - \mu_2)^2}\rho\sigma_1\sigma_2 + \frac{(\mu_1 - \mu)^2}{(\mu_1 - \mu_2)^2}\sigma_2^{2}
\end{equation}
Zauważmy, że zmienne $\sigma$ oraz $\mu$ zmieniają się ze zmianą wagi $w_1$, zaś zmienne związane z walorami (tzn. $\mu_1$, $\sigma_1$, $\mu_2$, $\sigma_2$, $\rho$) są niezależne od wag walorów. Równanie 3.8 można więc przekształcić w taki sposób, by utworzyć współczynnik $A$ dla zmiennych związanych z walorami:

\begin{equation}
	\sigma^2  = A(\mu - \mu_0)^2 + \sigma_0^2,
\end{equation}
gdzie:

\begin{equation}
	A = \frac{\sigma_1^{2} - 2\sigma_1\sigma_2\rho + \sigma_2^{2}}{(\mu_1 - \mu_2)^2} > 0,
\end{equation}
a $\sigma_0$ oraz $\mu_0$ to współrzędne określające wierzchołek hiperboli 3.10:

\begin{equation}
	\sigma_0^2 = \frac{\sigma_1^{2}\sigma_2^{2}(1-\rho^2)}{(\sigma_1 - \sigma_2)^2 + 2(1-\rho)\sigma_1\sigma_2} \ge 0,
\end{equation}

\begin{equation}
	\mu_0 = \frac{\mu_1\sigma_2^2 - (\mu_1 + \mu_2)\rho\sigma_1\sigma_2 + \mu_2\sigma_1^2}{(\sigma_1 - \sigma_2)^2 + 2(1-\rho)\sigma_1\sigma_2} \ge 0,
\end{equation}
gdzie w powyższych równaniach nierówność wynika z tego, że $ -1 < \rho < 1$. \cite{book}\\

Wykres krzywej opisanej równaniem 3.9 przedstawiony został na rysunku 3.2.
Otrzymano typowy kształt wyznaczający wcześniej wspomniane portfele wydajne - są to portfele z górnej części hiperboli. Zgodnie z teorią Markowitza dają one najwyższy spodziewany zwrot przy danym ryzyku. Portfele, których punkty znajdują się w dolnej części hiperboli nazywane są \textit{portfelami niewydajnymi} i nie są rozpatrywane w wyborze portfela. Jest to widoczne na rysunku 3.2 - ich oczekiwany zwrot jest niższy niż tych z górnej hiperboli przy jednakowym ryzyku.\\

Dla łatwiejszgo zrozumienia, oto przykład portfela będącego tematem tego podrodziału (to znaczy składającego się z dwóch aktywów ryzykownych). Zwrot portfela jest opisany zmienną losową $R$, a stopę zwrotu z inwestycji w aktywa po ustalonym okresie opisują zmienne $R_1$ oraz $R_2$. Wartości oczekiwane i odchylenia standardowe $R_1$ i $R_2$ wynoszą odpowiednio: $\mu_1 = 7\%$, $\sigma_1 = 5\%$, $\mu_2 = 12\%$, $\sigma_2 = 7\%$, a ich korelacja $\rho = 0,2$.\\
Aby znaleźć wierzchołek hiperboli opisującej ten portfel, podstawiamy dane odopwiednio do wzorów 3.11 i 3.12:
\begin{equation}
	\mu_0 = \frac{102}{12} = 8,5,
\end{equation}

\begin{equation}
	\sigma_0 = \sqrt{\sigma_0^2} = \sqrt{\frac{1176}{60}} \approx 4,43.
\end{equation}
Wierzchołek hiperboli znajduje się więc w punkcie o współrzędnych $(4,43; 8,5)$. Dane zaznaczono na rysunku 3.2:\\
\begin{figure}[h]
	\centering
	\includegraphics[scale=0.27]{wykres_markowitz_z_wierzcholkiem}
	\caption{Zależność odchylenia standardowego portfela i jego oczekiwanego zwrotu z zaznaczonym wierzchołkiem hiperboli.}
\end{figure}

Obliczenie wag aktywów dla tego przypadku jest teraz bardzo proste - podstawiając do wzoru z równania 3.7 wyliczony oczekiwany zwrot $\mu$ otrzymujemy wagę $w_1$, a do znalezienia $w_2$ korzystamy z równania 3.4:

\begin{equation}
	w_1 = \frac{8,5 - 12}{7 - 12} = 0,7,
\end{equation}

\begin{equation}
	w_2 = 1 - 0,7 = 0,3.
\end{equation}\newpage
Wynik ten oznacza, że jeżeli nasz budżet wynosi 1000 PLN powinniśmy zainwestować 700 PLN w aktywo 1, a 300 w aktywo 2, aby otrzymać maksymalny oczekiwany zwrot dla ryzyka $\sigma = 4,43$.\\

Jak wcześniej wspomniano, możliwy jest również scenariusz w którym jedno z aktywów będzie miało wagę ujemną i oznacza to, że powinniśmy takie aktywo sprzedać i za to i za kwotę, którą posiadamy kupić drugie aktywo. Jako że mamy ustalony budżet, sprzedaż taka jest w niego wliczona - tzn. mając 1000 PLN i wagi $w_1 = 1,2$ i $w_2 = -0,2$, odpowiedź brzmi: "Sprzedaj aktywo 2 o wartości 200 PLN i zainwestuj 1200 PLN w aktywo 1".\\

Inwestor może korzystać z różnych portfeli, balansując ryzykiem i stopą zwrotu, jednak trzeba pamiętać o założeniu Markowitza - inwestor jest racjonalny, a więc jego wybór ograniczony jest tylko i wyłącznie do portfeli z górnej części hiperboli, ponieważ dają one największy średni zysk przy danym ryzyku.

\section{Portfel optymalny dla danego inwestora}
W rozważaniach przyjęto portfel składający się z dwóch aktywów ryzykownych. Jak wiadomo, każdy człowiek jest inny - dlatego każdy inwestor ma pewną, określoną przez siebie tolerację na ryzyko wynikającą z różnych czynników. W teorii Markowitza ta tolerancja inwestora na ryzyko określana jest przez \textit{funkcję użyteczności} U. Jest to funkcja, której dziedziną są rozkłady stopy zwrotów. Zmienna losowa $R$ opisująca stopę zwrotu po danym okresie ma rozkład $F_R$. Funkcja użyteczności ma postać widoczną na równaniu 3.17:
\begin{equation}
U(F_R) = \mu - k\sigma^2,
\end{equation}
gdzie:\\
$\mu$ - średnia dystrybuanty $F_R$\\
$\sigma$ - odchylenie standardowe dystrybuanty $F_R$\\
$k$ - współczynnik awersji do ryzyka (indywidualny dla każdego inwestora)\\
Jak możemy zauważyć, do obliczeń potrzebne są tylko wartości $\mu$ oraz $\sigma$ tego rozkładu. Aby uzyskać portfel optymalny dla danego inwestora, należy znaleźć taki portfel wydajny (czyli, tak jak już wiadomo, leżący na górnej części hiperboli), który maksymalizuje funkcję użyteczności awersji do ryzyka.\\
W sensie analitycznym rozwiązanie przedstawia się następująco: szukamy maksimum funkcji U przy warunku 3.9 ograniczając się do górnej hiperboli tego równania, gdyż jak zostało wcześniej powiedziane - tylko te portfele są portfelami wydajnymi. Można pokazać, że szukany portfel optymalny to portfel wyznaczony przez punkt styczności funkcji użyteczności i górnej części hiperboli - nazwijmy go punktem $O$. Graficznie zostało to przedstawione na rysunku 3.3:\\
\newpage
\begin{figure}[h]
	\centering
	\includegraphics[scale=0.4]{wykres_markowitz_z_u}
	\caption{Wykres hiperboli wyznaczającej portfele oraz paraboli funkcji użyteczności U z zaznaczonym ich punktem wspólnym $O$.}
\end{figure}

Zadaniem jest znalezienie współrzędnych tego punktu. Aby nie używać oznaczeń $(\sigma_o, \mu_o)$, które mogą być mylące (O jest zbyt podobne do 0), użyto $(\sigma^*, \mu^*)$. Podstawiając $\sigma^2$ z równania 3.9 do równania 3.16:

\begin{equation}
	U = \mu^* - kA(\mu^* - \mu_0)^2 + k\sigma_0^2.
\end{equation}
Obliczany punkt maksymalny to taki, gdzie pochodna cząstkowa $\frac{\partial\\U}{\partial\mu^*} = 0$, więc:

\begin{equation}
	\frac{\partial\\U}{\partial\mu^*} = 1 - 2kA(\mu^* - \mu_0) = 0.
\end{equation}
Przekształćmy równanie, aby po jego lewej stronie znalazł się jedynie szukany oczekiwany zwrot portfela $\mu^*$:

\begin{equation}
	\mu^* = \mu_0 + \frac{1}{2Ak}.
\end{equation}
Odchylenie standardowe tego portfela obliczamy, podstawiając znalezione $\mu^*$ do równania hiperboli 3.9:

\begin{equation}
	\sigma^* = \sqrt{{\sigma_0}^2+\frac{1}{4Ak^2}}.
\end{equation}
Wyliczony z równań 3.20 i 3.21 punkt $(\sigma^*, \mu^* )$ stanowi punkt portfela optymalnego danego inwestora. Wagi poszczególnych aktywów dla tego portfela wyznacza się w analogiczny sposób, jak w sekcji 3.2 (korzystając z równania 3.8), oczywiście za $\mu$ przyjmując oczekiwany zwrot portfela optymalnego, czyli $\mu^*$.

\section{Portfele wydajne dla aktywa wolnego od ryzyka i jednego aktywa ryzykownego}
W portfelu może znajdować się \textit{aktywo wolne od ryzyka}. Jest to takie aktywo, którego stopa zwrotu przyjmuje jedną wartość z prawdopodobieństwem równym 1, a więc jego wariancja wynosi 0. Inwestor wybiera tylko jedno aktywo wolne od ryzyka, gdyż z założenia Markowitza o jego racjonalności, będzie wybierać te o najwyższej stopie zwrotu. Jak będzie wyglądać portfel wydajny zawierający takie aktywo? Załóżmy, że mamy portfel z dwoma aktywami, $S_1$ oraz $S_2$, gdzie $S_1$ jest aktywem ryzykownym, a $S_2$ wolnym od ryzyka. $R_1$ oraz $R_2$ to zmienne losowe opisujące stopę zwrotu z inwestycji w te aktywa, i są one opisane przez ich odchylenia standardowe $\sigma_1$ i $\sigma_2 = 0$ oraz wartości średnie $\mu_1$ i $r_f$. Znając wzór na odchylenie standardowe portfela (równanie 3.6) znajdźmy równanie opisujące oczekiwany zwrot z opisywanego portfela:

\begin{equation}
\sigma^2 = w_1^{2}\sigma_1^{2}.
\end{equation}
Podstawmy w miejsce $w_1$ wartość wyznaczoną w równaniu 3.7:

\begin{equation}
	\sigma = \frac{\mu - r_f}{\mu_1 - r_f}\sigma_1,
\end{equation}
po przekształceniu, aby po jego lewej stronie znalazł się jedynie szukany oczekiwany zwrot portfela $\mu$ otrzymujemy:

\begin{equation}
	\mu = r_f + \frac{\mu_1 - r_f}{\sigma_1}\sigma.
\end{equation}
Równanie 3.23, po przeniesieniu na wykres i podstawieniu danych ukazano na rysunku 3.4:

\begin{figure}[h]
	\centering
	\includegraphics[scale=0.27]{zero_risk_wykres}
	\caption{Zależność odchylenia standardowego portfela i jego oczekiwanego zwrotu dla portfela zawierającego aktywo ryzykowne ($\sigma_1, \mu_1$) oraz wolne od ryzyka ($0, r_f$).}
\end{figure}

Podsumowując - otrzymujemy, że portfele wydajne dla 2 aktywów, z których jedno jest wolne od ryzyka leżą na linii prostej przechodzącej przez współrzędne $(\sigma, \mu)$ tych aktywów.

\section{Portfele wydajne dla aktywa wolnego od ryzyka i dwóch aktywów ryzykownych}
W rozważaniach przyjęto portfel składający się z dwóch aktywów ryzykownych $S_1$ oraz $S_2$ oraz aktywa wolnego od ryzyka $S_3$. Zmienne losowe $R_1$, $R_2$ oraz $R_3$ opisują stopę zwrotu z inwestycji w te aktywa po ustalonym okresie. Zmienne $R_1$ oraz $R_2$ mają rozkład normalny i są opisane przez ich średnie $\mu$ i odchylenia standardowe $\sigma$. $R_3$ jest opisane przez stopę zwrotu $r_f$, a odchylenie standardowe tej zmiennej wynosi $\sigma_f = 0$, co wiemy z poprzedniego podrozdziału. Przyjmujemy również, że jedno z aktywów ma wyższą zarówno stopę zwrotu, jak i odchylenie standardowe ($\mu_1 < \mu_2$ oraz $\sigma_1 < \sigma_2$). Przykładowo, w sytuacji gdy $\mu_1 < \mu_2$ oraz $\sigma_1 > \sigma_2$ oczywistym byłby wybór aktywa opisanego przez zmienną losową $R_1$ - dawałoby ono większy oczekiwany zysk przy mniejszym ryzyku, a więc inwestowalibyśmy tylko w aktywo 2.\\
Wiemy, że portfele wydajne zawierające dwa aktywa ryzykowne leżą na górnej części hiperboli opisanej równaniem 3.9 oraz to, że portfele zawierające aktywo wolne od ryzyka i aktywo ryzykowne leżą na linii prostej opisanej równaniem 3.24. Opis graficzny z przykładem znajduje się na rysunku 3.5:\\ 

\begin{figure}[h]
	\centering
	\includegraphics[scale=0.4]{wykres_markowitz_proste_wydajne}
	\caption{Wykres hiperboli opisującej portfele aktywów ryzykownych oraz prostych opisujących portfele zawierające aktywo wolne od ryzyka.}
\end{figure}

Jak można zauważyć na rysunku 3.5, portfel leżący na prostej przechodzącej przez punkty $E$ oraz $(0, r_f)$ dla przykładowych danych ryzyk $\sigma_i$ oraz $\sigma_j$ daje wyższą stopę zwrotu (punkty $I$ oraz $J$) niż portfel leżący na prostej przechodzącej przez punkty $N$ oraz $(0, r_f)$ (punkty $I'$ oraz $J'$). Najlepszym portfelem - takim, który spełnia założenia Markowitza o maksymalizacji oczekiwanego zwrotu oraz minimalizacji ryzyka - będzie więc portfel leżący na prostej stycznej do hiperboli, a więc zawierającej punkt $E$. Portfel z punktu $E$, w tym wypadku składujący się z 2 aktywów ryzykownych $S_1$ i $S_2$, nazywamy \textit{portfelem styczności}. Portfele wydajne leżą na prostej przechodzącej przez punkt $(0,r_f)$ i punkt $E$, tak więc składają się z aktywa wolnego od ryzyka i portfela $E$. Stąd każdy inwestor niezależnie od swojej awersji do ryzyka ma taki sam portfel aktywów ryzykownych, portfele różnią jedynie się częścią zainwestowaną w aktywo wolne od ryzyka i portfel styczności $E$. Nazywamy to \textit{twierdzeniem o wzajemnej separacji funduszy} (ang. \textit{mutual fund separation theorem})\\ 

Było to odkrycie, które wywołało szeroką dyskusję w świecie finansów. W niedługim czasie po Markowitzu zaczęto rozwijać tę ideę, czego skutkiem były między innymi opublikowana przez Jamesa Tobina w 1958 praca na temat \textit{super-efektywnego portfela}. Poźniej, w 1964 Williama Sharpe wydał pracę w której opracował \textit{model wyceny aktywów kapitałowych - (ang. CAPM - Capital Asset Pricing Model)}\cite{holton}\\

Zajmijmy się problemem analitycznego wyznaczenia portfela styczności. Poszukiwane są współrzędne portfela styczności (tzn. punktu $E$ na rysunku 3.5) oraz jego wagi. Jest to przypadek, w którym żadne środki nie są przeznaczone na aktywo wolne od ryzyka, a więc $w_f = 0$. Przekształćmy równania z podpunktu 3.2, aby znajdowały się w nich 3 aktywa: 

\begin{equation}
	w_1 + w_2 + w_f = 1,
\end{equation}
\begin{equation}
	\mu = w_1\mu_1 + w_2\mu_2 + w_fr_f,
\end{equation}
\begin{equation}
	\sigma^2 = w_1^2\sigma_1^2 + 2\sigma_{12}w_1w_2 + w_2^2\sigma_2^2,
\end{equation}
gdzie:\\
$\sigma_{12}$ - kowariancja zmiennych losowych $R_1$ oraz $R_2$.\\
Celem będzie znalezienie minimalnej wartości $\sigma$ przy warunku 3.25 i 3.26. Postępując analogicznie, jak w sekcji 3.2, na początek możemy pozbyć się jednej niewiadomej z równania opisującego wagi - niech będzie to $w_f$:

\begin{equation}
	w_f = 1 - w_1 - w_2,
\end{equation}
Podstawiając otrzymaną wagę $w_f$ do równania 3.26 można pozbyć się jednego równania, a więc celem będzie znalezienie minimalnej wartości $\sigma$ przy spełnieniu tylko jednego założenia - równania 3.29:

\begin{equation}
	\mu - r_f = (\mu_1 - r_f)w_1 + (\mu_2 - r_f)w_2.
\end{equation}
Aby znaleźć minimum funkcji $\sigma$, użyjemy mnożników Lagrange'a\cite{lagrange}. W tym celu, aby uzyskać funkcję ograniczającą w odpowiedniej dla mnożników Lagrange'a postaci $g(w_1, w_2) = 0$, przekształamy równanie 3.29 do następującej postaci:

\begin{equation}
g(w_1, w_2) = (\mu_1 - r_f)w_1 + (\mu_2 - r_f)w_2 - \mu + r_f = 0.
\end{equation}
Stąd:

\begin{equation}
	\nabla\\g = (\frac{\partial\\g}{w_1}, \frac{\partial\\g}{w_2}) = (\mu_1 - r_f, \mu_2 - r_f),
\end{equation}

\begin{equation}
	\nabla\sigma_p = (\frac{\partial\sigma}{w_1}, \frac{\partial\sigma}{w_2}) = (2{\sigma_1}^2w_1 + 2\sigma_{12}w_2, 2{\sigma_2}^2w_2 + 2\sigma_{12}w_1),
\end{equation}
Z metody mnożników Langrange'a otrzymujemy:

\begin{equation}
	\left( \begin{array}{cc}
		{\sigma_1}^2 & \sigma_{12}\\
		\sigma_{12} & {\sigma_2}^2
	\end{array} \right)
	\left( \begin{array}{c}
		z_1\\
		z_2
	\end{array} \right) =
	\left( \begin{array}{c}
		\mu_1 - r_f\\
		\mu_2 - r_f
	\end{array} \right),
\end{equation}
Pierwszym człon równania 3.33 to \textit{macierz wariancji-kowariancji}. Zawiera ona kowariancje wszystkich aktywów ryzykownych występujących w portfelu, a na jej przekątnej znajdują się wariancje tych aktywów. Nie może ona być określona dowolnie - jest ona zawsze kwadratowa, symetryczna, a jej wyznacznik jest nieujemny.
Zmienne $z_1$ oraz $z_2$ w równaniu 3.33 są zmiennymi pomocniczymi, które wyznaczają wagi $w_j$ portfela (równania 3.34 oraz 3.35):

\begin{equation}
	w_1 = \frac{z_1}{z_1 + z_2},
\end{equation}
\begin{equation}
	w_2 = \frac{z_2}{z_1 + z_2},
\end{equation}
Współrzędne portfela styczności wynoszą zatem:

\begin{equation}
	\mu_T = w_1\mu_1 + w_2\mu_2,
\end{equation}

\begin{equation}
	\sigma_T = \sqrt{w_1^{2}\sigma_1^{2} + w_2^{2}\sigma_2^{2} + 2w_1w_2\sigma_{12}},
\end{equation}
za $w_1$ oraz $w_2$ podstawiając wartości wyliczone odpowiednio równaniami 3.34 i 3.35\cite{book}. Pamiętajmy, że wagi $w_1$ oraz $w_2$ są tutaj wagami aktywów ryzykownych - nie będą się więc one zmieniać w stosunku do siebie w zależności od inwestora.\\

Warto również wyznaczyć równanie prostej portfeli wydajnych, służy do tego równanie prostej przechodzącej przez dwa punkty:

\begin{equation}
	y = y_1 + \frac{y_2 - y_1}{x_2 - x_1}(x - x_1),
\end{equation}
do którego podstawiamy współrzędne punktu $(0, r_f)$ oraz otrzymane współrzędne portfela styczności $(\sigma_T, \mu_T)$:

\begin{equation}
	\mu = r_f + \frac{\mu_T - r_f}{\sigma_T}\sigma.
\end{equation}
\newpage
\section{Portfel optymalny dla danego inwestora}
W rozważaniach przyjęto portfel składający się z dwóch aktywów ryzykownych oraz aktywa wolnego od ryzyka taki sam jak w podrozdziale 3.5. Chcąc w takim przypadku znaleźć portfel optymalny dla danego inwestora, należy znaleźć taki portfel wydajny (tzn. leżący na prostej portfeli wydajnych), który maksymalizuje funkcję użyteczności awersji do ryzyka. Można graficznie pokazać, że portfel optymalny dla naszego inwestora graficznie jest portfelem $O$ na rysunku 3.6:\\

\begin{figure}[h]
	\centering
	\includegraphics[scale=0.4]{wykres_markowitz_ryzyk_z_u}
	\caption{Wykres paraboli funkcji użyteczności danego inwestora U z zaznaczonym punktem styczności $O$ do prostej opisującej portfele wydajne zawierające aktywo wolne od ryzyka.}
\end{figure}

Znając współczynnik awersji do ryzyka $k$ dla danego inwestora oraz mając wyznaczone równanie prostej portfeli wydajnych z równania 3.39, portfel znajdujemy, korzystając z wyznaczonych w podpunkcie 3.3 równań 3.21 i 3.22:

\begin{equation}
	\mu^* = \mu_0 + \frac{1}{2Ak},
\end{equation}

\begin{equation}
	\sigma^* = \sqrt{{\sigma_0}^2+\frac{1}{4Ak^2}}.
\end{equation}
W równaniach tych za $A$, $\sigma_0$ oraz $\mu_0$ podstawiamy wartości wyliczone odpowiednio za pomocą równań 3.10, 3.11 oraz 3.12, gdzie:
\begin{itemize}
	\item Za wartości $\mu_1$, $\sigma_1$ podstawiamy dane aktywa wolnego od ryzyka, a więc $r_f$, $0$
	\item Za wartości $\mu_2$, $\sigma_2$ podstawiamy dane portfela styczności, a więc $\mu_T$, $\sigma_T$
	\item Korelacja $\rho = 1$
\end{itemize} 
Otrzymany punkt o współrzędnych ($\mu^*, \sigma^*$) stanowi punkt portfela optymalnego danego inwestora. Wagi wyznaczamy również analogicznie, jak w podpunkcie 3.3 - przy czym trzeba pamiętać, że i one mają w tym przypadku inne znaczenie - są to wagi inwestycji w aktywo wolne od ryzyka oraz w portfel aktywów ryzykownych (traktowany jako jedno aktywo) - stąd otrzymujemy z równania 3.7:

\begin{equation}
w_1 = \frac{\mu^* - \mu_T}{r_f - \mu_T},
\end{equation}

Zatem, przykładowo interpretacja wyniku przy budżecie 100 PLN oraz wagach $w_1 = 65\%, w_2 = 35\%$ brzmiałaby następująco: "Zainwestuj 65 PLN w aktywo wolne od ryzyka, a 35 PLN zainwestuj w aktywa ryzykowne w proporcjach wyznaczonych w równaniach 3.34 oraz 3.35"

\section{Aktualizacja portfela przy zmianie stopy wolnej od ryzyka}
W rozważaniach przyjęto portfel składający się z dwóch aktywów ryzykownych oraz aktywa wolnego od ryzyka taki sam jak w sekcji 3.5. Załóżmy, że chcemy w szybki sposób uaktualniać portfel styczności, gdy zmienia się stopa zwrotu z aktywa wolnego od ryzyka.
W takim przypadku można przekształcić równanie 3.33 przez przemnożenie obu jego stron przez macierz wariancji-kowariancji. Wynikiem tego działania jest:

\begin{equation}
	\left( \begin{array}{c}
		z_1\\
		z_2\\
	\end{array} \right) = 
	\left( \begin{array}{cc}
		c_{11}& c_{12}\\
		c_{21}& c_{22}\\
	\end{array} \right)
	\left( \begin{array}{c}
		1\\
		r_f\\
	\end{array} \right),
\end{equation}
gdzie macierz zawierająca wiersze $c_{nm}$ zawiera stałe pomocnicze, które zależą jedynie od aktywów ryzykownych. Sposób ten jest na tyle wygodny, że posiadając wartości macierzy zmiennych $z$ dla dwóch różnych wartości $r_f$, można z układu równań wyznaczyć macierz wierszy $c_{nm}$ i od tej pory szybko aktualizować wagi portfela w zależności od stopy zwrotu aktywa wolnego od ryzyka. Macierz wierszy $c_{nm}$ zawsze posiada dwie kolumny, a ilość wierszy odpowiada ilości aktywów ryzykownych w portfelu. W odróżnieniu od macierzy wariancji-kowariancji, ta macierz nie musi być symetryczna.

\section{Portfel styczności - dowolna liczba aktywów ryzykownych}
W tym punkcie w rozważaniach przyjęto $n$ aktywów ryzykownych w portfelu oraz jedno aktywo bez ryzyka o stopie zwrotu $r_f$. Zadaniem jest znalezienie portfela styczności, podobnie jak w sekcji 3.5. Wzory wyglądają więc podobnie, jednak tym razem posiadamy $n$ aktywów:

\begin{equation}
	\mu = w_1\mu_1 + w_2\mu_2 + ... + w_n\mu_n + w_fr_f,
\end{equation}

\begin{equation}
	w_1 + w_2 + \dots + w_n + w_f = 1.
\end{equation}
Podobnie jak poprzednio korzystając z metody mnożników Lagrange'a otrzymujemy:

\begin{equation}
	\left( \begin{array}{cccc}
		{\sigma_1}^2& \sigma_{12}  & \dots & \sigma_{1n}\\
		\sigma_{12} & {\sigma_2}^2 & \dots & \sigma_{2n}\\
		\vdots 		& \vdots 	   &       & \vdots  \\
		\sigma_{n1} & \sigma_{n2}  & \dots & \sigma_{n}^2\\
	\end{array} \right)
	\left( \begin{array}{c}
		z_1\\
		z_2\\
		\vdots\\
		z_n
	\end{array} \right) =
	\left( \begin{array}{c}
		\mu_1 - r_f\\
		\mu_2 - r_f\\
		\vdots\\
		\mu_n - r_f
	\end{array} \right),
\end{equation}
gdzie poszczególne wagi aktywów ryzykownych portfela styczności oblicza się analogicznie jak w podpunkcie 3.5:

\begin{equation}
w_j = \frac{z_j}{z_1 + \dots + z_n},
\end{equation}
gdzie $j = \{1, 2, 3, \dots, n\}$\\
Można pokazać, że kolumnę zmiennych $z_j$ można przedstawić jako kolumnę ${\lambda}w_j$:

\begin{equation}
	\left( \begin{array}{cccc}
		{\sigma_1}^2& \sigma_{12}  & \dots & \sigma_{1n}\\
		\sigma_{12} & {\sigma_2}^2 & \dots & \sigma_{2n}\\
		\vdots 		& \vdots 	   &       & \vdots  \\
		\sigma_{n1} & \sigma_{n2}  & \dots & \sigma_{n}^2\\
	\end{array} \right)
	\left( \begin{array}{c}
		{\lambda}w_1\\
		{\lambda}w_2\\
		\vdots\\
		{\lambda}w_n
	\end{array} \right) =
	\left( \begin{array}{c}
		\mu_1 - r_f\\
		\mu_2 - r_f\\
		\vdots\\
		\mu_n - r_f
	\end{array} \right),
\end{equation}
gdzie:\\ 
$\lambda$ - stosunek zysku do zmienności\\
$w_j$ - waga aktywa j\\
Stosunek zysku do zmienności $\lambda$ można zapisać wzorem:

\begin{equation}
	\lambda = \frac{\mu - r_f}{{\sigma}^2},
\end{equation}
gdzie $\mu$ oraz $\sigma$ to współrzędne naszego portfela styczności.\\

W przypadku zmieniającej się stopy zwrotu z aktywa wolnego od ryzyka, podobnie jak w sekcji 3.7 możemy wyznaczyć macierz zmiennych pomocniczych $c_{nm}$, w tym przypadku posiadającej $n$ wierszy:

\begin{equation}
	\left( \begin{array}{c}
		z_1\\
		z_2\\
		\vdots\\
		z_n
	\end{array} \right) = 
	\left( \begin{array}{cc}
		c_{11}& c_{12}\\
		c_{21}& c_{22}\\
		\vdots& \vdots\\
		c_{n1}& c_{n2}\\
	\end{array} \right)
	\left( \begin{array}{c}
		1\\
		r_f\\
	\end{array} \right)
\end{equation}

\chapter{Założenia projektowe}

\begin{itemize}
	\item Opcja zapisu stanu portfela do pliku oraz jego wczytanie z pliku.
	\item Umożliwienie działania programu w dwóch trybach:
	\begin{itemize}
		\item Dane o aktywach znajdujących się w portfelu pobierane z internetu (za pomocą \textit{Yahoo Finance API})
		\item Dane o aktywach znajdujących się w portfelu pobierane z pliku tekstowego
	\end{itemize}
	\item Możliwość ustalenia budżetu użytkownika
	\item Wyznaczanie portfela styczności
	\item Możliwość zapisu współczynnika awersji do ryzyka użytkownika
	\item Wyznaczanie portfela optymalnego dla użytkownika
	\item Możliwość określenia horyzontu czasowego, z jakiego mają zostać pobrane wyniki
	\item Możliwość wyboru, z ilu dni mają być liczone zwroty
\end{itemize}

\chapter{Prezentacja oraz opis aplikacji}

Na podstawie założeń projektowych powstało narzędzie wyznaczające optymalny portfel inwestora, jego nazwa to \code{the-wallet}. Pełen kod programu został umieszczony w dodatku A, jest on również dostępny do pobrania na stronie GitHub. Narzędzie zostało napisane w języku Python, w zgodności z wersją 3.7.8. Składa się ono z trzech plików: \code{main.py}, \code{functions.py} oraz \code{additional\_functions.py}, których funkcjonalność zostanie omówiona odpowiednich podrozdziałach tej części pracy. Na początek zostanie zaprezentowane działanie programu.

\section{Przykładowe uruchomienie programu w trybie API}

Program zgodnie z założeniami projektowymi można uruchomić w dwóch trybach - w tym podrozdziale zostanie omówiony tryb pobierania danych przez API (dalej nazywanego w skrócie trybem API). Pobierane dane są danymi rzeczywistymi - możemy znaleźć je na stronie \textit{Yahoo Finance}. Aby dokonać uruchomienia programu w trybie API znajdując się w katalogu roboczym narzędzia należy wywołać komendę:
\mint{bash}{python3 main.py}, po udanym uruchomieniu programu ukazuje on użytkownikowi główne menu. Poprzez wpisanie odpowiedniego numeru dokonujemy wyboru interesującej nas funkcji.

\begin{figure}[ht]
	\centering
	\includegraphics[scale=0.44]{interfejs1}
	\caption{Główne menu programu}
\end{figure}
Załóżmy, że chcemy stworzyć nowy portfel. W tym celu wybieramy opcję numer 2, która powoduje przejście do menu portfela (zostało to uwiecznione na rysunku 5.2).
\newpage
\begin{figure}[ht]
	\centering
	\includegraphics[scale=0.44]{interfejs2}
	\caption{Menu portfela}
\end{figure}
Nasz portfel jest w tym momencie pusty - nie posiadamy żadnych aktywów ryzykownych ani aktywa wolnego od ryzyka. Dodajmy dwa aktywa ryzykowne - niech będą to aktywa Apple'a oraz Facebooka. Aby dodać aktywa, należy podać ich symbole po wybraniu opcji numer 1. Po sprawdzeniu na stronie \textit{https://finance.yahoo.com/} możemy zobaczyć, że symbolami tych aktyw są odpowiednio kody AAPL i FB. Wynik operacji dodania aktywów do portfela widoczny jest na rysunkach 5.3a oraz 5.3b.

\begin{figure}[hb]
	\centering
	\subfloat[Dodanie do portfela aktyw firmy Apple]{\label{aapl}
	\includegraphics[width=0.4\textwidth]{apple}}
	\quad
	\subfloat[Dodanie do portfela aktyw firmy Facebook]{\label{fb}
	\includegraphics[width=0.4\textwidth]{fb}}
	\caption{Dodanie aktyw ryzykownych do portfela}
\end{figure}

Następnym krokiem w tworzeniu portfela będzie dodanie aktywa wolnego od ryzyka - w tym celu należy wybrać opcję 3 i podać roczną stopę zwrotu takiego aktywa - w tym przypadku niech będzie to $2\%$.
\newpage
\begin{figure}[ht]
	\centering
	\includegraphics[scale=0.4]{rfassetapi}
	\caption{Dodanie do portfela aktywa wolnego od ryzyka}
\end{figure}

Dla poprawnego działania narzędzia portfel musi posiadać zdefiniowaną funkcję użyteczności awersji od ryzyka. W tym celu po wybraniu opcji 4 podajemy współczynnik $k$ awersji do ryzyka inwestora - niech wynosi on $k=0,1$.

\begin{figure}[ht]
	\centering
	\includegraphics[scale=0.4]{kapi}
	\caption{Dodanie do portfela współczynnika $k$ awersji do ryzyka inwestora}
\end{figure}

Pozostałe zmienialne parametry w portfelu mają swoje następujące wartości domyślne, nadawane w momencie tworzenia portfela:

\begin{itemize}
	\item zakres dni z których rozpatrujemy dane wynosi 500 dni (w trybie API),
	\item domyślna średnia cena aktywów jest średnią jednodniową,
	\item budżet wynosi 100 USD,
\end{itemize}
Na potrzeby obecnego uruchomienia nie będziemy ich zmieniać. Sprawdźmy jak wygląda w tej chwili portfel. Umożliwia to kryjąca się pod numerem $0$ opcja \code{Show wallet}. Gotowy portfel ukazano na rysunku 5.6.
\newpage
\begin{figure}[ht]
	\centering
	\includegraphics[scale=0.3]{showwalletapi}
	\caption{Gotowy portfel w trybie API}
\end{figure}

Zobaczmy, jak wygląda portfel optymalny dla wprowadzonych danych - w tym celu używamy opcji \code{Show optimal portfolio weights}. Nie będziemy w tym momencie interpretować tego wyniku (na interpretacje przeznaczony jest rozdział szósty tej pracy). Wyniki przedstawiono na rysunku 5.7.

\begin{figure}[ht]
	\centering
	\includegraphics[scale=0.4]{optimalportfolioapi}
	\caption{Przykładowy optymalny portfel w trybie API}
\end{figure}
\newpage

\section{Przykładowe uruchomienie programu w trybie file-mode}

Uruchomienie w trybie pobierania danych z pliku tekstowego (dalej nazywanego w skrócie \textit{file-mode}) wymaga od nas odpowiedniego przygotowania takowego pliku. Przykład odpowiednio sformatowania został przedstawiony w poniższym bloku kodu:
\begin{minted}{bash}
	Stock1;Stock2;Stock3
	2.22;2.34;3.71
	2.22;2.35;3.78
	2.24;2.32;3.79
	2.19;2.37;3.81
	2.24;2.38;3.81
	2.25;2.40;3.82
	2.26;2.41;3.84
	2.22;2.40;3.87
\end{minted}
W pierwszym wierszu należy podać oddzielone średnikami nazwy spółek, których kursy zamknięcia będą rozpatrywane w narzędziu. Następnie, w kolejnych wierszach należy podawać chronologicznie od najstarszego do najnowszego dzienne kursy zamknięcia dla spółek, tworząc w ten sposób swego rodzaju tabelę. Komenda uruchomienia programu w opisywanym trybie wygląda następująco:
\mint{bash}{python3 main.py -f dane.txt},
gdzie dane.txt jest ścieżką do plikiu z zapisanymi danymi. Oczywiście nazwa \code{dane.txt} jest przykładowa - plik może nazywać się dowolnie i mieć dowolne rozszerzenie.
Na potrzeby przykładu przygotowano dane, które zapisano do pliku tekstowego o nazwie \code{mydata.txt}, jak ukazano na rysunku 5.8. 

\begin{figure}[ht]
	\centering
	\includegraphics[scale=0.4]{catmydata}
	\caption{Przygotowany plik tekstowy \code{mydata.txt} do uruchomienia w trybie \textit{file-mode}}
\end{figure}

Następnie uruchamiamy program - jak możemy zauważyć na rysunku 5.9, jesteśmy przywitani komunikatem o tym, że narzędzie działa w trybie \textit{file-mode} i od razu zostajemy przeniesieni do menu portfela.

\begin{figure}[ht]
	\centering
	\includegraphics[scale=0.4]{interfejs2filemode}
	\caption{Uruchomienie narzędzia w trybie \textit{file-mode}}
\end{figure}

\newpage
Dane do portfela wprowadzamy dokładnie w taki sam sposób jak w trybie API. Na potrzeby tego przykładu ustalamy poziom rocznego zwrotu z aktywa wolnego od ryzyka na poziomie $2 \%$, a współczynnika $k$ awersji do ryzyka inwestora na wartości $k=0,7$. Dla reszty parametrów pozostawiamy ich domyślne wartości. Gotowy portfel został przedstawiony na rysunku 5.10.

\begin{figure}[ht]
	\centering
	\includegraphics[scale=0.37]{showwalletfilemode}
	\caption{Gotowy portfel w trybie \textit{file-mode}}
\end{figure}
Jak możemy zauważyć, program dostosował zakres dni z których rozpatrujemy dane do długości wprowadzonych danych - wprowadziliśmy 8 cen, więc domyślną wartością zakresu jest teraz 8 dni.
Sprawdźmy, jak wygląda portfel optymalny dla wprowadzonych danych. Wynik znajduje się na rysunku 5.11.
\newpage
\begin{figure}[ht]
	\centering
	\includegraphics[scale=0.4]{optimalportfoliofilemode}
	\caption{Przykładowy optymalny portfel w trybie \textit{file-mode}}
\end{figure}

Wprowadzony portfel możemy zapisać do pliku, aby móc z niego korzystać przy kolejnych uruchomieniach programu. W tym celu wybieramy opcję \code{Save wallet to file} a następnie wprowadzamy nazwę pliku do którego chcemy zapisać nasz portfel. Odczytać go możemy wybierając przy kolejnym uruchomieniu opcję \textit{Open existing wallet}. Cały opisany proces został przedstawiony na rysunku 5.12.

\begin{figure}[h]
	\centering
	\includegraphics[scale=0.342]{openwlt}
	\caption{Zapis portfela do pliku oraz odczytanie portfela}
\end{figure}
\newpage
\section{main.py - opis funkcjonalności}
Jest to plik za pomocą którego uruchamiany jest program, zawiera ona import zewnętrznej biblioteki \code{argparse}, wewnętrznej \code{functions} oraz jedną funkcję - \code{main} - przedstawioną na rysunku 5.13.

\begin{figure}[ht]
	\centering
	\includegraphics[scale=0.37]{main}
	\caption{Funkcja \code{main}}
\end{figure}

Funkcja ta ma dwa zadania: przetworzenie opcjonalnego argumentu \code{-f} do uruchomienia trybu \textit{file-mode} oraz uruchomienie interfejsu programu, znajdującego się w pliku \code{functions.py}.\\


\section{functions.py oraz additional\_functions.py - opis funkcjonalności}

Plik \code{functions.py} jest trzonem działania programu - znajdują się z nim dwie główne klasy: \code{Wallet} i \code{Interface} oraz jedna pomocnicza \code{Stock}.
Klasa pomocnicza \code{Stock} jest używana tylko podczas uruchomionego trybu \textit{file-mode} i symuluje ona strukturę danych pobranych za pomocą \textit{Yahoo Finance API}, dzięki czemu działanie funkcji programu dla obu trybów działania (poza funkcją \code{get\_stocks\_returns}) jest takie samo.
\begin{figure}[ht]
	\centering
	\includegraphics[scale=0.37]{classstock}
	\caption{Klasa \code{Stock}}
\end{figure}

Klasę \code{Wallet} ukazano na rysunku 5.15. Przechowuje ona zmienne związane bezpośrednio z portfelem oraz wyniki kalkulacji wykonywanych w celu obliczenia portfela optymalnego dla inwestora. Oprócz tego, posiada swoją definicję funkcji \code{\_\_str\_\_}, dzięki której użycie w programie formuły \code{print} na obiekcie klasy \code{Wallet} zwraca wszystkie główne informacje dotyczące obecnego stanu portfela (wynik uruchomienia możemy zobaczyć na rysunkach 5.6 oraz 5.10).
\newpage
\begin{figure}[ht]
	\centering
	\includegraphics[scale=0.37]{classwallet}
	\caption{Klasa \code{Wallet}}
\end{figure}

Kolejna składowa pliku to klasa \code{Interface} - jej obiekt jest tworzony w pliku \code{main.py}. Funkcja inicjalizacyjna obiektu tej klasy najpierw sprawdza, w jakim trybie zostało uruchomione narzędzie, a następnie wyświetla główne menu i przekierowuje do menu portfela odpowiedniego dla wybranego trybu działania programu.

\begin{figure}[ht]
	\centering
	\includegraphics[scale=0.4]{initinterface}
	\caption{Funkcja inicjalizacyjna obiektu klasy \code{Interface}}
\end{figure}

Funkcja \code{main\_menu} klasy \code{Interface} pozwala na wybór jednej z trzech opcji: odczytania portfelu z pliku, otwarcia nowego portfela (tzn. utworzenie obiektu klasy \code{Wallet} w obiekcie klasy \code{Interface}) lub opuszczenia programu. W przypadku uruchomienia narzędzia w trybie \textit{file-mode} menu to jest pomijane - zostaje wyświetlony komunikat o włączonym trybie \textit{file-mode} i funkcja kończy swoje działanie. Menu jest zabezpieczone przed przypadkiem wpisania w strumieniu wejściowym czegokolwiek poza dozwolonymi symbolami (czyli na przykład w tym przypadku cyfry \code{1}, \code{2} lub \code{3}) przez komunikat o nieprawidłowo wybranej funkcji i powrót do możliwości wpisania prawidłowego symbolu.

\begin{figure}[ht]
	\centering
	\includegraphics[scale=0.37]{mainmenucode}
	\caption{Funkcja \code{main\_menu} klasy \code{Interface}}
\end{figure}
\newpage
Widoczna w funkcji \code{main\_menu} funkcja \code{load\_stocks\_info\_from\_file} (ukazana na rysunku 5.18) w pierwszej kolejności tworzy obiekt klasy \code{Wallet} w obiekcie klasy \code{Interface} - jest to portfel na którym będziemy operować w tym uruchomieniu programu. Następnie funkcja zamienia strumień wejściowy (czyli dane z naszego pliku) na strukturę \code{DataFrame} z biblioteki \code{pandas}. Następuje sprawdzenie, czy plik jest odpowiednio sformatowany oraz czy oprócz nazw spółek i separatorów \code{;} zawiera jedynie dane numeryczne (jeżeli tak nie jest, działanie programu zostaje przerwane) - poźniej dane są zapisywane do listy aktywów ryzykownych \code{stocks} w portfelu jako obiekty klasy pomocnicznej \code{Stock}.

\begin{figure}[ht]
	\centering
	\includegraphics[scale=0.37]{loadfromfilecode}
	\caption{Funkcja \code{load\_stocks\_info\_from\_file} klasy \code{Interface}}
\end{figure}

Po wybraniu opcji otworzenia portfela z pliku uruchamiana jest funkcja \code{open\_existing} \code{\_wallet}. W pierwszej kolejności prosi nas o wpisanie ścieżki do pliku, gdzie znajduje się portfel na którym chcemy działać w tym uruchomieniu programu. Jeżeli zarówno ścieżka jak i portfel istnieją, obiekt portfela jest tworzony z obiektu zapisanego do pliku za pomocą biblioteki \code{pickle}. Jeżeli otwierany przez nas portfel był tworzony w trybie \textit{file-mode}, wyświetlany jest o tym odpowiedni komunikat. Funkcja działa w pętli do momentu wpisania odpowiedniej ścieżki zawierającej prawidłowo zapisany portfel.

\begin{figure}[ht]
	\centering
	\includegraphics[scale=0.37]{openfromfilecode}
	\caption{Funkcja \code{open\_existing\_wallet} klasy \code{Interface}}
\end{figure}
\newpage
Przejdźmy do opisu menu portfela, które posiada dwie wersje - \code{wallet\_menu\_api} oraz \newline
\code{wallet\_menu\_file\_mode}. W omawianiu funkcji skupimy się na tym pierwszym, gdyż jest to po prostu wzbogacona wersja menu portfela w trybie \textit{file-mode}, a działanie dostępnych funkcji w obu trybach jest takie samo dzięki pomocniczej klasie \code{Stock}.
Funkcja \code{wallet\_menu\_api} posiada do wyboru jedenaście opcji, wybieranych przez wpisanie z klawiatury opowiedniej cyfry od \code{0} do \code{10}. Podobnie jak \code{main\_menu}, obecne jest zabezpieczenie przed przypadkiem wpisania czegokolwiek poza dozwolonymi symbolami. 

\begin{figure}[ht]
	\centering
	\includegraphics[scale=0.37]{walletmenucode}
	\caption{Funkcja \code{wallet\_menu\_api} klasy \code{Interface}}
\end{figure}

Funkcje \code{add\_stock\_to\_wallet} oraz \code{remove\_stock\_from\_wallet} są dostępne jedynie w trybie \textit{API}. Umożliwiają modyfikację listy posiadanych aktywów ryzykownych w portfelu. Funkcja dodająca aktywo do portfela przyjmuje od użytkownika symbol aktywa, pod którym jest ono dostępne na \textit{Yahoo Finance}. Następnie sprawdza, czy symbol ten istnieje w bazie - jeżeli tak, aktywo jest dodawane do portfela. W przeciwnym przypadku zwracany jest komunikat o błędzie.
Funkcja usuwająca aktywo z portfela również przyjmuje od użytkownika symbol aktywa, którego chce się pozbyć. Obecność aktywa w portfelu jest weryfikowana i w przypadku istnienia aktywa w portfelu jest ono usuwane.

\begin{figure}[ht]
	\centering
	\includegraphics[scale=0.37]{addremovestockcode}
	\caption{Funkcje \code{add\_stock\_to\_wallet} oraz \code{remove\_stock\_from\_wallet} klasy \code{Interface}}
\end{figure}

Opcja pod numerem \code{3} uruchamia funkcję \code{update\_risk\_factor}, aktualizująca współczynnik awersji do ryzyka inwestora. Jako że nie obowiązują żadne wytyczne co do zakresu w którym powinna się znajdować jego wartość, program jedynie upewnia się że wprowadzona przez użytkownika wartość jest możliwa do zapisu jako zmienna typu \textit{float} (liczba zmiennoprzecinkowa).

\begin{figure}[ht]
	\centering
	\includegraphics[scale=0.37]{updatekcode}
	\caption{Funkcja \code{update\_risk\_factor} klasy \code{Interface}}
\end{figure}

Funkcja \code{update\_risk\_free\_asset} przyjmuje roczny zwrot z aktywa wolnego od ryzyka w postaci procentowej, po czym zamienia go na liczbę. Wpisywana wartość musi znajdować się w przedziale od 0 do 100, w innym przypadku zwracany jest błąd. Funkcja działa w pętli do momentu wpisania wartości spełniającej te warunki.

\begin{figure}[ht]
	\centering
	\includegraphics[scale=0.37]{updaterfcode}
	\caption{Funkcja \code{update\_risk\_free\_asset} klasy \code{Interface}}
\end{figure}
\newpage
Funkcja \code{update\_amount\_of\_days\_to\_take\_data\_from} aktualizuje zakres dni z których rozpatrujemy dane dotyczące aktywów ryzykownych. Domyślna wartość wynosi 500 dni w trybie API oraz $n$ dni dla danych z $n$ dni podanych w trybie \textit{file-mode}. Funkcja przyjmuje od użytkownika ilość dni, a więc liczbę całkowitą. Liczba wpisana przez użytkownika musi:
\begin{itemize}
	\item być nieujemna,
	\item być mniejsza niż zakres dostępnych danych, co jest sprawdzane przez funkcję \code{given\_da} \code{ta\_is\_long\_enough} importowaną z pliku \code{additional\_functions.py} (warunek sprawdzany tylko w trybie \textit{file-mode}).
\end{itemize}
Funkcja działa w pętli do momentu wpisania wartości spełniającej te warunki.

\begin{figure}[ht]
	\centering
	\includegraphics[scale=0.37]{updatedayscode}
	\caption{Funkcja \code{update\_amount\_of\_days\_to\_take\_data\_from} klasy \code{Interface}}
\end{figure}

\begin{figure}[ht]
	\centering
	\includegraphics[scale=0.37]{givendatalongenoughcode}
	\caption{Funkcja \code{given\_data\_is\_long\_enough}}
\end{figure}

Funkcja \code{update\_return\_calculation} aktualizuje ilość dni z których liczona jest średnia cena zamknięcia aktywów ryzykownych. Domyślna wartość tej zmiennej wynosi 1 dzień. Funkcja przyjmuje od użytkownika ilość dni, a więc liczbę całkowitą. Liczba wpisana przez użytkownika w obu funkcjach musi:
\begin{itemize}
	\item być nieujemna,
	\item być mniejsza niż zakres dostępnych danych (warunek sprawdzany tylko w trybie \textit{file-mode}),
	\item być mniejsza lub równa zakresowi dni z których rozpatrujemy dane dotyczące aktywów ryzykownych.
\end{itemize}
Funkcja działa do momentu wpisania wartości spełniającej te warunki.

\begin{figure}[ht]
	\centering
	\includegraphics[scale=0.37]{updateretdayscode}
	\caption{Funkcja \code{update\_return\_calculation} klasy \code{Interface}}
\end{figure}
\newpage
Funkcja \code{update\_budget} aktualizuje budżet, dla którego będą liczone kwoty inwestycji w dane aktywa. Domyślna wartość to 100 USD. Wpisana przez użytkownika wartość musi być większa niż zero i być liczbą całkowitą. Funkcja działa do momentu wpisania wartości spełniającej te warunki.
\begin{figure}[ht]
	\centering
	\includegraphics[scale=0.37]{updatebudgetcode}
	\caption{Funkcja \code{update\_budget} klasy \code{Interface}}
\end{figure}

Najbardziej rozbudowanym elementem programu jest funkcja \code{show\_optimal\_portfol} \code{io\_weights}. Na początek przelicza ona wprowadzony roczny zwrot z aktywa wolnego od ryzyka na okres, z jakiego liczony jest średni zwrot z aktywa ryzykownego (tzn. dla średnich jednodniowych wprowadzony roczny zwrot z aktywa wolnego od ryzyka jest przeliczany na dzienny zwrot). Następnie wyliczany jest portfel styczności, a więc wagi aktywów ryzykownych w portfelu. Są one wyświetlane za pomocą funkcji \code{show\_risk\_assets\_weights}. W następnej kolejności wyświetlana jest macierz kowariancji (jako jedyna jest wyświetlana zwykłą komendą \code{print} dla celów estetyczniejszego sposobu ukazania macierzy), oraz liczone są wagi portfela optymalnego. Ostatnim elementem funkcji jest wyświetlenie użytkownikowi zgodnie z wprowadzonym przez niego budżetem, jakie kwoty powinien on zainwestować w dane aktywa.

\begin{figure}[ht]
	\centering
	\includegraphics[scale=0.37]{showoptimalweightscode}
	\caption{Funkcja \code{show\_optimal\_portfolio\_weights} klasy \code{Interface}}
\end{figure}

\begin{figure}[h]
	\centering
	\includegraphics[scale=0.37]{showriskweightscode}
	\caption{Funkcja \code{show\_risk\_assets\_weights} klasy \code{Interface}}
\end{figure}

\begin{figure}[h]
	\centering
	\includegraphics[scale=0.37]{showcovmartixcode}
	\caption{Funkcja \code{show\_covariation\_matrix} klasy \code{Interface}}
\end{figure}
\newpage
Funkcja \code{calculate\_tangent\_portfolio\_weights} na początku oblicza zwroty z aktywów ryzykownych znajdujących w naszym portfelu. Jeżeli wybrany tryb działania to \textit{API}, za pomocą funkcji \code{history} obiektu klasy \code{Ticker} z biblioteki \code{yfinance} pobierane są dane aktywa ryzykownego i zapisywane w obiekcie klasy \code{DataFrame}. Dla trybu \textit{file-mode} dane pobierane są z listy \textit{self.wallet.stocks} która została stworzona podczas działania funkcji \code{load\_stocks\_info\_from\_file}  Zwrot liczony jest przez obliczenie zmian procentowych wartości z kolumny \code{['Adj Close']}, czyli ceny zamknięcia z uwzględnionymi przypadkami dywidendy oraz podziału akcji - drugie z tych obliczeń jest dokonywane przez funkcji \code{apply\_stock\_splits} importowaną z pliku \code{additional\_functions.py}. Zwroty te zapisywane są w obiekcie klasy \code{DataFrame}.

\begin{figure}[ht]
	\centering
	\includegraphics[scale=0.37]{calculatetangentcode}
	\caption{Funkcja \code{calculate\_tangent\_portfolio\_weights} klasy \code{Interface}}
\end{figure}

\begin{figure}[ht]
	\centering
	\includegraphics[scale=0.37]{getstocksreturnscode}
	\caption{Funkcja \code{get\_stocks\_returns} klasy \code{Interface}}
\end{figure}

Następnie z pomocą funkcjonalności \textit{list comprehension} języka \textit{Python}\cite{pythonlist} oraz funkcji \code{calculate\_z\_matrix} obliczane są zmienne pomocnicze \textit{z} (ich rola jest wyjaśniona w podrozdziale 3.5). Są one przeliczane na wagi aktywów ryzykownych w portfelu również za pomocą \textit{list comprehension}.

\begin{figure}[ht]
	\centering
	\includegraphics[scale=0.37]{calczmatrixcode}
	\caption{Funkcja \code{calculate\_z\_matrix} klasy \code{Interface}}
\end{figure}
\newpage
Przejdźmy do omówienia funkcji \code{calculate\_optimal\_portfolio\_weights}. Na początku oblicza ona parametry portfela styczności - to znaczy jego oczekiwanego zwrotu i odchylenia standardowego. Nastepnie obliczane są parametry portfela optymalnego - oczekiwany zwrot, odchylenie standardowe oraz \textit{A} (tak jak zostało to wyjaśnione w podrozdziale 3.6). Wyliczone parametry pozwalają nam osiągnąć ostateczny cel programu - obliczenie wagi aktywa ryzykownego oraz portfela aktywów ryzykownych. Sformatowane do przyjaznego użytkownikowi formatu wyniki są przedstawiane za pomocą funkcji \code{show\_budget\_calculations}. Po zakończeniu działania funkcji program wraca do menu portfela.

\begin{figure}[ht]
	\centering
	\includegraphics[scale=0.37]{calcoptportwcode}
	\caption{Funkcja \code{calculate\_optimal\_portfolio\_weights} klasy \code{Interface}}
\end{figure}

\begin{figure}[h]
	\centering
	\includegraphics[scale=0.3]{gettanportparamscode}
	\caption{Funkcja \code{get\_tangent\_portfolio\_parameters} klasy \code{Interface}}
\end{figure}

\begin{figure}[h]
	\centering
	\includegraphics[scale=0.3]{getoptportparamscode}
	\caption{Funkcja \code{get\_optimal\_portfolio\_parameters} klasy \code{Interface}}
\end{figure}

\begin{figure}[h]
	\centering
	\includegraphics[scale=0.3]{showbudgetcalccode}
	\caption{Funkcja \code{show\_budget\_calculations} klasy \code{Interface}}
\end{figure}

\newpage
Funkcja \code{save\_wallet\_to\_file} umożliwia zapis aktualnie otwartego portfela do pliku. W tym celu użytkownik wpisuje ścieżkę, pod którą chciałby zapisać portfel. Jeżeli plik pod wskazaną ścieżką już istnieje, użytkownik stawiany jest przed wyborem nadpisania istniejącego już pliku aktualnym portfelem albo wybrania innej ścieżki. Zapis odbywa się za pomocą funkcji \code{dump} z biblioteki \code{pickle}. Funkcja działa do momentu poprawnego zapisu portfela.

\begin{figure}[h]
	\centering
	\includegraphics[scale=0.3]{savewalletcode}
	\caption{Funkcja \code{save\_wallet\_to\_file} klasy \code{Interface}}
\end{figure}
\chapter{Ocena działania programu}

Test działania 
aplikacji w rzeczywistości - (np zebranie wyników, inwestycja “na sucho” i analiza po miesiącu.)


\chapter{Wnioski oraz plany rozwoju aplikacji}

\appendixpage
\appendix
%\addappheadtotoc

\chapter{Dodatek - pełny kod programu}\label{Dod1}

\section{main.py}
\inputminted[breaklines]{python}{the-wallet/main.py}

\section{functions.py}
\inputminted[breaklines]{python}{the-wallet/functions.py}

\section{additional\_functions.py}
\inputminted[breaklines]{python}{the-wallet/additional_functions.py}

\bibliographystyle{dyplom1}

\begin{thebibliography}{9}
	\bibitem{random_variable}
	Wikipedia,
	\emph{Zmienna losowa},
	https://pl.wikipedia.org/wiki/Zmienna\_losowa
	
	\bibitem{expected_value}
	Wikipedia,
	\emph{Wartość oczekiwana},
	https://pl.wikipedia.org/wiki/Wartość\_oczekiwana
	
	\bibitem{wikipage}
	Wikipedia,
	\emph{Teoria Portfelowa}.
	https://pl.wikipedia.org/wiki/Teoria\_portfelowa
	
	\bibitem{book}	
	Michael A. Bean,
	\emph{Probability: The Science of Uncertainty: with Applications to Investments, Insurance, and Engineering}.
	American Mathematical Society, 2001.
	
	\bibitem{lagrange}
	Uniwersytet Warszawski,
	\emph{Ekstrema Warunkowe i Mnożniki Lagrange'a}.
	http://dydmat.mimuw.edu.pl/analiza-matematyczna-ii/ekstrema-warunkowe-i-mnozniki-lagrangea
	
	\bibitem{holton}
	Glyn Holton,
	\emph{Portfolio Theory}.
	https://www.glynholton.com/notes/portfolio\_theory/,
	2013
	
	\bibitem{pythonlist}
	Python Software Foundation,
	\emph{Documentation -> The Python Tutorial -> Data Structures}
	https://docs.python.org/3/tutorial/datastructures.html
	
\end{thebibliography}
\begin{comment}
Można więc stworzyć układ równań:

$$
\left\{ \begin{array}{l}
2{\sigma_1}^2w_1 + 2\sigma_{12}w_2 - \tau(\mu_1 - r_f) = 0\\
2{\sigma_2}^2w_2 + 2\sigma_{12}w_1 - \tau(\mu_2 - r_f) = 0\\
(\mu_1 - r_f)w_1 + (\mu_2 - r_f)w_2 - \mu_p + r_f = 0
\end{array} \right.
$$

Przekształcając pierwsze dwa równania otrzymujemy:

$$
\left\{ \begin{array}{l}
{\sigma_1}^2w_1 + \sigma_{12}w_2 = \frac{\tau}{2}(\mu_1 - r_f)\\
{\sigma_2}^2w_2 + \sigma_{12}w_1 = \frac{\tau}{2}(\mu_2 - r_f)\\
(\mu_1 - r_f)w_1 + (\mu_2 - r_f)w_2 - \mu_p + r_f = 0
\end{array} \right.
$$



\begin{equation}
\left( \begin{array}{cc}
{\sigma_1}^2 & \sigma_{12}\\
\sigma_{12} & {\sigma_2}^2
\end{array} \right)
\left( \begin{array}{c}
w_1\\
w_2
\end{array} \right) =
(\frac{\tau}{2})
\left( \begin{array}{c}
\mu_1 - r_f\\
\mu_2 - r_f
\end{array} \right)
\end{equation}
\end{comment}
\end{document}
